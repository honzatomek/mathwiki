\documentclass[10pt,b5paper,titlepage]{book}

\usepackage[utf8]{inputenc}
\usepackage{amsmath}
\usepackage{amsfonts}
\usepackage{amssymb}
%\usepackage{xcolor}
\usepackage{color}
\usepackage{graphicx}

% python - needs pip install Pygments
\usepackage{minted}
\usemintedstyle{native}

\usepackage{hyperref}

\author{Jan Tomek}
\title{\bf Playground}

% python code format
\setminted[python]{breaklines, framesep=2mm, fontsize=\footnotesize, numbersep=5pt}
\newminted[python]{python}{linenos=true,
                           frame=lines,
                           baselinestretch=1.2,
                           mathescape,
                           xleftmargin=0cm,
                           framesep=2mm,
                           fontsize=\footnotesize}

\setlength{\parindent}{0ex}
\setlength{\parskip}{1em plus 0.1em minus 0.2em}
\renewcommand{\labelitemi}{$\bullet$}
\renewcommand{\labelitemii}{$\bullet$}
\renewcommand{\labelitemiii}{$\bullet$}
\renewcommand{\labelitemiv}{$\bullet$}

%Commands definitions
\newcommand\setbackgroundcolour{\pagecolor[rgb]{0.15,0.15,0.15}}
\newcommand\settextcolour{\color[rgb]{0.9,0.9,0.9}}
\newcommand\invertbackgroundtext{\setbackgroundcolour\settextcolour}

\newcommand*{\Scale}[2][4]{\scalebox{#1}{$#2$}}%
\newcommand*{\Resize}[2]{\resizebox{#1}{!}{$#2$}}%

% Lagrangian symbol
\newcommand{\lagr}{\mathop{\mathcal{L}}}
\DeclareMathOperator{\Lagr}{\mathcal{L}}

% matrix notation
\newcommand{\m}{\mathbf}
% matrix notation greek letters
\newcommand{\M}{\pmb}

% null symbol
\newcommand{\Null}{\text{\O}}

\newcommand{\closure}[2][3]{%
{}\mkern#1mu\overline{\mkern-#1mu#2}}

% maximum number of entries in matrix
\setcounter{MaxMatrixCols}{20}

% equal by definition
\newcommand*\eqd{\stackrel{\triangle}{=}}

% {name}[number of arguments][1st default value] etc..
% the argument value is then inserted at #argument_number
\newenvironment{bbox}[1][0.96]
{
    \begin{center}
        \begin{tabular}{|p{#1\textwidth}|}
            \hline\\
}
{
            \\\\\hline
        \end{tabular}
    \end{center}
}

\newenvironment{bboxtitle}[1][0.96]
{
    \begin{center}
        #1\\[1ex]
        \begin{tabular}{|p{#1\textwidth}|}
            \hline\\
}
{
            \\\\\hline
        \end{tabular}
    \end{center}
}

\newcount\myloopcounter
\newcommand{\repeatit}[3][10]{%
    \myloopcounter1% initialize the loop counter
    \loop\ifnum\myloopcounter < #1
    #2#3%
    \advance\myloopcounter by 1%
    \repeat% start again
    #2%
}

\newenvironment{qbox}
{
%\centering{\huge{?}}
\begin{center}
    \repeatit[44]{?}{\ }
\end{center}
%\hrule
}
{
%\hrule
\begin{center}
    \repeatit[44]{?`}{\ }
\end{center}
}

\newenvironment{eqarray}
{
    \begin{eqnarray}
        \begin{aligned}
}
{
        \end{aligned}
    \end{eqnarray}
}

%Command execution.
%If this line is commented, then the appearance remains as usual.
\invertbackgroundtext

\begin{document}

\maketitle

\chapter{RBE2}

\begin{bbox}[0.95]
    The rigid freebody relations between \textbf{slave} and \textbf{master} node are:
    \begin{eqarray}
        \m{T}_s &= \m{T}_m + \m{R}_s \times \overline{\m{x}} \\
        \m{R}_s &= \m{R}_m
    \end{eqarray}

    Where $ \overline{\m{x}} $ is a vector from \textbf{slave} to \textbf{master}:
    \begin{equation}
        \overline{\m{x}} =
        \begin{bmatrix}
            \overline{x} \\
            \overline{y} \\
            \overline{z}
        \end{bmatrix} =
        \begin{bmatrix}
            x_s - x_m \\
            y_s - y_m \\
            z_s - z_m
        \end{bmatrix}
    \end{equation}

    Then \textit{vector multiplication} (cross product) is defined as:
    \begin{eqarray}
        \m{R}_s \times \m{x}_s
        &= \m{R}_m \times \m{x} = & \\
        &= \begin{bmatrix}
             \phi_m \\
             \psi_m \\
             \theta_m
           \end{bmatrix} \times \begin{bmatrix}
             x_s - x_m \\
             y_s - y_m \\
             z_s - z_m
           \end{bmatrix}
        &= \begin{bmatrix}
             \phi_m \\
             \psi_m \\
             \theta_m
           \end{bmatrix} \times \begin{bmatrix}
             \overline{x} \\
             \overline{y} \\
             \overline{z}
           \end{bmatrix} = \\
        &= \begin{bmatrix}
             \psi_m (z_s - z_m ) - \theta_m (y_s - y_m) \\
             \theta_m (x_s - x_m) - \phi_m (z_s - z_m) \\
             \phi_m (y_s - y_m) - \psi_m (x_s - x_m)
           \end{bmatrix}
        &= \begin{bmatrix}
            \psi_m \overline{z} - \theta_m \overline{y} \\
            \theta_m \overline{x} - \phi_m \overline{z} \\
            \phi_m \overline{y} - \psi_m \overline{x}
        \end{bmatrix} = \\
        &= \begin{bmatrix}
            \phantom{-}0 & \phantom{-}\overline{z} & -\overline{y} \\
            -\overline{z} & \phantom{-}0 & \phantom{-}\overline{x} \\
            \phantom{-}\overline{y} & -\overline{x} & \phantom{-}0
        \end{bmatrix}
        \begin{bmatrix}
            \phi_m \\
            \psi_m \\
            \theta_m
        \end{bmatrix}
    \end{eqarray}
\end{bbox}

\begin{bbox}[0.95]
    The \textbf{slave} equations can be written as:
    \begin{equation}
        \begin{bmatrix}
            u_s \\
            v_s \\
            w_s \\
            \phi_s \\
            \psi_s \\
            \theta_s
        \end{bmatrix}
        = \begin{bmatrix}
            1 & 0 & 0 & 0 & \phantom{-(}z_s - z_m & -(y_s - y_m) \\
            0 & 1 & 0 & -(z_s - z_m) & 0 & \phantom{-(}x_s - x_m \\
            0 & 0 & 1 & \phantom{-(}y_s - y_m & -(x_s - x_m) & 0 \\
            0 & 0 & 0 & 1 & 0 & 0 \\
            0 & 0 & 0 & 0 & 1 & 0 \\
            0 & 0 & 0 & 0 & 0 & 1 \\
        \end{bmatrix}
        \begin{bmatrix}
            u_m \\
            v_m \\
            w_m \\
            \phi_m \\
            \psi_m \\
            \theta_m
        \end{bmatrix}
    \end{equation}

    or:
    \begin{equation}
        \begin{bmatrix}
            u_s \\
            v_s \\
            w_s \\
            \phi_s \\
            \psi_s \\
            \theta_s
        \end{bmatrix}
        = \begin{bmatrix}
            1 & 0 & 0 & \phantom{-}0 & \phantom{-}\overline{z} & -\overline{y} \\
            0 & 1 & 0 & -\overline{z} & \phantom{-}0 & \phantom{-}\overline{x} \\
            0 & 0 & 1 & \phantom{-}\overline{y} & -\overline{x} & \phantom{-}0 \\
            0 & 0 & 0 & \phantom{-}1 & \phantom{-}0 & \phantom{-}0 \\
            0 & 0 & 0 & \phantom{-}0 & \phantom{-}1 & \phantom{-}0 \\
            0 & 0 & 0 & \phantom{-}0 & \phantom{-}0 & \phantom{-}1 \\
        \end{bmatrix}
        \begin{bmatrix}
            u_m \\
            v_m \\
            w_m \\
            \phi_m \\
            \psi_m \\
            \theta_m
        \end{bmatrix}
    \end{equation}

    Finally the force and moment conditions have to be imposed:
    \begin{eqarray}
        \m{F}_m + \sum_{i=1}^n \m{F}_{s,i} &= \Null \\
        \m{M}_m + \sum_{i=1}^n \m{M}_{s,i} &= \Null \\
    \end{eqarray}

    which is equal to:
    \begin{eqarray}
        k_{m,u} u_m + \sum_{i=1}^n k_{s,u,i} u_{s,i} &= \Null \\
        k_{m,v} v_m + \sum_{i=1}^n k_{s,v,i} v_{s,i} &= \Null \\
        k_{m,w} w_m + \sum_{i=1}^n k_{s,w,i} w_{s,i} &= \Null \\
        k_{m,\phi} \phi_m + \sum_{i=1}^n
        \left(
            k_{s,\phi,i} \phi_{s,i}
            + k_{s,v,i} v_{s,i} \overline{z}_{i}
            + k_{s,w,i} w_{s,i} \overline{y}_{i}
    \right) &= \Null \\
        k_{m,\psi} \psi_m + \sum_{i=1}^n
        \left(
            k_{s,\psi,i} \psi_{s,i}
            + k_{s,u,i} u_{s,i} \overline{z}_{i}
            + k_{s,w,i} w_{s,i} \overline{x}_{i}
        \right) &= \Null \\
        k_{m,\theta} \theta_m + \sum_{i=1}^n
        \left(
            k_{s,\theta,i} \theta_{s,i}
            + k_{s,u,i} u_{s,i} \overline{y}_{i}
            + k_{s,v,i} v_{s,i} \overline{x}_{i}
        \right) &= \Null \\
    \end{eqarray}

\end{bbox}

\section{RBE2 Example:}

Master Node:
\begin{equation}
    \m{X}_m =
    \begin{bmatrix}
        1 & 1 & 1
    \end{bmatrix}^T
\end{equation}

Slave Nodes:
\begin{eqarray}
    \m{X}_{s,1} =
    \begin{bmatrix}
        0 & 0 & 0
    \end{bmatrix}^T \\
    \m{X}_{s,1} =
    \begin{bmatrix}
        2 & 0 & 0
    \end{bmatrix}^T \\
    \m{X}_{s,1} =
    \begin{bmatrix}
        2 & 2 & 0
    \end{bmatrix}^T \\
    \m{X}_{s,1} =
    \begin{bmatrix}
        0 & 2 & 0
    \end{bmatrix}^T \\
\end{eqarray}

Then Slave Node 1 equations are:
\begin{eqarray}
    \m{x}_1 &=
    \begin{bmatrix}
        x_{s,1} - x_m & y_{s,1} - y_m & z_{s,1} - z_m
    \end{bmatrix}^T =
    \begin{bmatrix}
        -1 & -1 & -1
    \end{bmatrix}^T
\end{eqarray}

Other Slave Nodes:
\begin{eqarray}
    \m{x}_2 &=
    \begin{bmatrix}
        2 - 1 & 0 - 1 & 0 - 1
    \end{bmatrix}^T =
    \begin{bmatrix}
        \phantom{-}1 & -1 & -1
    \end{bmatrix}^T \\
    \m{x}_3 &=
    \begin{bmatrix}
        2 - 1 & 2 - 1 & 0 - 1
    \end{bmatrix}^T =
    \begin{bmatrix}
        \phantom{-}1 & \phantom{-}1 & -1
    \end{bmatrix}^T \\
    \m{x}_4 &=
    \begin{bmatrix}
        0 - 1 & 2 - 1 & 0 - 1
    \end{bmatrix}^T =
    \begin{bmatrix}
        -1 & \phantom{-}1 & -1
    \end{bmatrix}^T
\end{eqarray}

Therefore, after filling in the coefficients:
\renewcommand\arraystretch{1.6}
\begin{eqarray}
    \begin{bmatrix}
        u_{s,1} \\
        v_{s,1} \\
        w_{s,1} \\
        \phi_{s,1} \\
        \psi_{s,1} \\
        \theta_{s,1}
    \end{bmatrix}
    &= \begin{bmatrix}
        1 & 0 & 0 & \phantom{-}0 & -1 & \phantom{-}1 \\
        0 & 1 & 0 & \phantom{-}1 & \phantom{-}0 & -1 \\
        0 & 0 & 1 & -1 & \phantom{-}1 & \phantom{-}0 \\
        0 & 0 & 0 & \phantom{-}1 & \phantom{-}0 & \phantom{-}0 \\
        0 & 0 & 0 & \phantom{-}0 & \phantom{-}1 & \phantom{-}0 \\
        0 & 0 & 0 & \phantom{-}0 & \phantom{-}0 & \phantom{-}1 \\
    \end{bmatrix}
    \begin{bmatrix}
        u_m \\
        v_m \\
        w_m \\
        \phi_m \\
        \psi_m \\
        \theta_m
    \end{bmatrix} \\
    \begin{bmatrix}
        u_{s,2} \\
        v_{s,2} \\
        w_{s,2} \\
        \phi_{s,2} \\
        \psi_{s,2} \\
        \theta_{s,2}
    \end{bmatrix}
    &= \begin{bmatrix}
        1 & 0 & 0 & \phantom{-}0 & -1 & \phantom{-}1 \\
        0 & 1 & 0 & \phantom{-}1 & \phantom{-}0 & \phantom{-}1 \\
        0 & 0 & 1 & -1 & -1 & \phantom{-}0 \\
        0 & 0 & 0 & \phantom{-}1 & \phantom{-}0 & \phantom{-}0 \\
        0 & 0 & 0 & \phantom{-}0 & \phantom{-}1 & \phantom{-}0 \\
        0 & 0 & 0 & \phantom{-}0 & \phantom{-}0 & \phantom{-}1 \\
    \end{bmatrix}
    \begin{bmatrix}
        u_m \\
        v_m \\
        w_m \\
        \phi_m \\
        \psi_m \\
        \theta_m
    \end{bmatrix} \\
    \begin{bmatrix}
        u_{s,3} \\
        v_{s,3} \\
        w_{s,3} \\
        \phi_{s,3} \\
        \psi_{s,3} \\
        \theta_{s,3}
    \end{bmatrix}
    &= \begin{bmatrix}
        1 & 0 & 0 & \phantom{-}0 & -1 & -1 \\
        0 & 1 & 0 & \phantom{-}1 & \phantom{-}0 & \phantom{-}1 \\
        0 & 0 & 1 & \phantom{-}1 & -1 & \phantom{-}0 \\
        0 & 0 & 0 & \phantom{-}1 & \phantom{-}0 & \phantom{-}0 \\
        0 & 0 & 0 & \phantom{-}0 & \phantom{-}1 & \phantom{-}0 \\
        0 & 0 & 0 & \phantom{-}0 & \phantom{-}0 & \phantom{-}1 \\
    \end{bmatrix}
    \begin{bmatrix}
        u_m \\
        v_m \\
        w_m \\
        \phi_m \\
        \psi_m \\
        \theta_m
    \end{bmatrix} \\
    \begin{bmatrix}
        u_{s,4} \\
        v_{s,4} \\
        w_{s,4} \\
        \phi_{s,4} \\
        \psi_{s,4} \\
        \theta_{s,4}
    \end{bmatrix}
    &= \begin{bmatrix}
        1 & 0 & 0 & \phantom{-}0 & -1 & -1 \\
        0 & 1 & 0 & \phantom{-}1 & \phantom{-}0 & -1 \\
        0 & 0 & 1 & \phantom{-}1 & \phantom{-}1 & \phantom{-}0 \\
        0 & 0 & 0 & \phantom{-}1 & \phantom{-}0 & \phantom{-}0 \\
        0 & 0 & 0 & \phantom{-}0 & \phantom{-}1 & \phantom{-}0 \\
        0 & 0 & 0 & \phantom{-}0 & \phantom{-}0 & \phantom{-}1 \\
    \end{bmatrix}
    \begin{bmatrix}
        u_m \\
        v_m \\
        w_m \\
        \phi_m \\
        \psi_m \\
        \theta_m
    \end{bmatrix}
\end{eqarray}
\renewcommand\arraystretch{1}

When all put together:
\begin{equation}
    \begin{bmatrix}
        u_{s,1} \\
        v_{s,1} \\
        w_{s,1} \\
        \phi_{s,1} \\
        \psi_{s,1} \\
        \theta_{s,1} \\
        u_{s,2} \\
        v_{s,2} \\
        w_{s,2} \\
        \phi_{s,2} \\
        \psi_{s,2} \\
        \theta_{s,2} \\
        u_{s,3} \\
        v_{s,3} \\
        w_{s,3} \\
        \phi_{s,3} \\
        \psi_{s,3} \\
        \theta_{s,3} \\
        u_{s,4} \\
        v_{s,4} \\
        w_{s,4} \\
        \phi_{s,4} \\
        \psi_{s,4} \\
        \theta_{s,4}
    \end{bmatrix}
    = \begin{bmatrix}
        1 & 0 & 0 & \phantom{-}0 & -1 & \phantom{-}1 \\
        0 & 1 & 0 & \phantom{-}1 & \phantom{-}0 & -1 \\
        0 & 0 & 1 & -1 & \phantom{-}1 & \phantom{-}0 \\
        0 & 0 & 0 & \phantom{-}1 & \phantom{-}0 & \phantom{-}0 \\
        0 & 0 & 0 & \phantom{-}0 & \phantom{-}1 & \phantom{-}0 \\
        0 & 0 & 0 & \phantom{-}0 & \phantom{-}0 & \phantom{-}1 \\
        1 & 0 & 0 & \phantom{-}0 & -1 & \phantom{-}1 \\
        0 & 1 & 0 & \phantom{-}1 & \phantom{-}0 & \phantom{-}1 \\
        0 & 0 & 1 & -1 & -1 & \phantom{-}0 \\
        0 & 0 & 0 & \phantom{-}1 & \phantom{-}0 & \phantom{-}0 \\
        0 & 0 & 0 & \phantom{-}0 & \phantom{-}1 & \phantom{-}0 \\
        0 & 0 & 0 & \phantom{-}0 & \phantom{-}0 & \phantom{-}1 \\
        1 & 0 & 0 & \phantom{-}0 & -1 & -1 \\
        0 & 1 & 0 & \phantom{-}1 & \phantom{-}0 & \phantom{-}1 \\
        0 & 0 & 1 & \phantom{-}1 & -1 & \phantom{-}0 \\
        0 & 0 & 0 & \phantom{-}1 & \phantom{-}0 & \phantom{-}0 \\
        0 & 0 & 0 & \phantom{-}0 & \phantom{-}1 & \phantom{-}0 \\
        0 & 0 & 0 & \phantom{-}0 & \phantom{-}0 & \phantom{-}1 \\
        1 & 0 & 0 & \phantom{-}0 & -1 & -1 \\
        0 & 1 & 0 & \phantom{-}1 & \phantom{-}0 & -1 \\
        0 & 0 & 1 & \phantom{-}1 & \phantom{-}1 & \phantom{-}0 \\
        0 & 0 & 0 & \phantom{-}1 & \phantom{-}0 & \phantom{-}0 \\
        0 & 0 & 0 & \phantom{-}0 & \phantom{-}1 & \phantom{-}0 \\
        0 & 0 & 0 & \phantom{-}0 & \phantom{-}0 & \phantom{-}1 \\
    \end{bmatrix}
    \begin{bmatrix}
        u_m \\
        v_m \\
        w_m \\
        \phi_m \\
        \psi_m \\
        \theta_m
    \end{bmatrix}
\end{equation}

It can be read as:
\begin{eqarray}
    u_{s,1} &= u_m -\psi_m + \theta_m \\
    v_{s,1} &= v_m \phi_m - \theta_m \\
    \vdots \\
    \theta_{s,4} &= \theta_m
\end{eqarray}

Or:
\begin{equation}\label{rbe2-coefficients}
    \m{u}_s = \m{\closure{T}} \m{u}_m
\end{equation}

This equation should be then expanded to all DOFs, where in $ \m{T} $ the
slave DOFs columns are missing and the DOFs that do not take part in RBEs have $ 1 $ on
the diagonal. The resulting  matrix should have the same number of rows as
the master stiffness matrix.

Then the master stiffness equation is modified:
\begin{eqarray}\label{rbe2-master-slave-modified}
    \m{\hat{K}} &= \m{T}^T \m{K} \m{T} \\
    \m{\hat{f}} &= \m{T}^T \m{f} \\
    \m{\hat{K}} \m{\hat{u}} &= \m{\hat{f}} \\
\end{eqarray}

If we'd like to make the procedure more general, partition the master stiffness
equations such that:
\begin{equation}
    \begin{bmatrix}
        \m{K}_{uu} & \m{K}_{um} & \m{K}_{us} \\
        \m{K}_{um}^T & \m{K}_{mm} & \m{K}_{ms} \\
        \m{K}_{us}^T & \m{K}_{ms}^T & \m{K}_{ss}
    \end{bmatrix}
    \begin{bmatrix}
        \m{u}_u \\
        \m{u}_m \\
        \m{u}_s
    \end{bmatrix}
    = \begin{bmatrix}
        \m{f}_u \\
        \m{f}_m \\
        \m{f}_s
    \end{bmatrix}
\end{equation}

\begin{qbox}
    Based on \eqref{rbe2-master-slave-modified} we could probably partition
    the matrices as:
    \begin{equation}\label{rbe2-partition-T-u-m-s}
        \begin{matrix}
            \begin{matrix}
                \begin{matrix}
                    \ 
                \end{matrix} \\
                \left. \begin{matrix}
                    u \\
                    m \\
                    s \\
                \end{matrix} \right|
            \end{matrix}
            & \begin{matrix}
                \begin{matrix}
                    u & m \\
                    \hline
                \end{matrix} \\
                \begin{bmatrix}
                    \m{I} & \m{0} \\
                    \m{0} & \m{I} \\
                    \m{0} & \m{\closure{T}} \\
                \end{bmatrix} \\
            \end{matrix}
        \end{matrix}
    \end{equation}

    If we expand \eqref{rbe2-coefficients} and partition the matrices
    based on \eqref{rbe2-partition-T-u-m-s} into independent or
    \textbf{unconstrained}, \textbf{master}
    and \textbf{slave} nodes, then the equation:
    \begin{equation}
        \m{u} = \m{T} \m{\hat{u}}
    \end{equation}

    becomes:
    \begin{equation}
        \begin{bmatrix}
            \m{u}_u  \\
            \m{u}_m \\
            \m{u}_s \\
        \end{bmatrix}
        = \begin{bmatrix}
            \m{I} & \m{0} \\
            \m{0} & \m{I} \\
            \m{0} & \m{\closure{T}} \\
        \end{bmatrix}
        \begin{bmatrix}
            \m{u}_u \\
            \m{u}_m \\
        \end{bmatrix}
    \end{equation}


    Finally the equation \eqref{rbe2-master-slave-modified} can be rewritten as:
    \begin{equation}
        \begin{bmatrix}
            \m{I} & \m{0} & \m{0} \\
            \m{0} & \m{I} & \m{\closure{T}}^T \\
        \end{bmatrix}
        \begin{bmatrix}
            \m{K}_{uu} & \m{K}_{um} & \m{K}_{us} \\
            \m{K}_{mu} & \m{K}_{mm} & \m{K}_{ms} \\
            \m{K}_{su} & \m{K}_{sm} & \m{K}_{ss} \\
        \end{bmatrix}
        \begin{bmatrix}
            \m{I} & \m{0} \\
            \m{0} & \m{I} \\
            \m{0} & \m{\closure{T}} \\
        \end{bmatrix}
        \begin{bmatrix}
            \m{u}_u \\
            \m{u}_m \\
        \end{bmatrix}
        = \begin{bmatrix}
            \m{I} & \m{0} & \m{0} \\
            \m{0} & \m{I} & \m{\closure{T}}^T \\
        \end{bmatrix}
        \begin{bmatrix}
            \m{f}_u \\
            \m{f}_m \\
            \m{f}_s \\
        \end{bmatrix}
    \end{equation}

    Expanding and multiplying:
    \begin{equation}
        \scalebox{0.92}{
        $ \begin{bmatrix}
            \m{K}_{uu} & \m{K}_{um} & \m{K}_{us} \\
            \m{K}_{mu} + \m{\closure{T}}^T \m{K}_{su} &
            \m{K}_{mm} + \m{\closure{T}}^T \m{K}_{sm} &
            \m{K}_{ms} + \m{\closure{T}}^T \m{K}_{ss} \\
        \end{bmatrix}
        \begin{bmatrix}
            \m{I} & \m{0} \\
            \m{0} & \m{I} \\
            \m{0} & \m{\closure{T}} \\
        \end{bmatrix}
        \begin{bmatrix}
            \m{u}_u \\
            \m{u}_m \\
        \end{bmatrix}
        = \begin{bmatrix}
            \m{f}_u \\
            \m{f}_m + \m{\closure{T}}^T \m{f}_s \\
        \end{bmatrix} $
        }
    \end{equation}

    \begin{equation}
        \scalebox{0.95}{
        $ \begin{bmatrix}
            \m{K}_{uu} & \m{K}_{um} + \m{K}_{us} \m{\closure{T}} \\
            \m{K}_{mu} + \m{\closure{T}}^T \m{K}_{su} &
            \m{K}_{mm} + \m{\closure{T}}^T \m{K}_{sm} +
            \m{K}_{ms} \m{\closure{T}} + \m{\closure{T}}^T \m{K}_{ss} \m{\closure{T}} \\
        \end{bmatrix}
        \begin{bmatrix}
            \m{u}_u \\
            \m{u}_m \\
        \end{bmatrix}
        = \begin{bmatrix}
            \m{f}_u \\
            \m{f}_m + \m{\closure{T}}^T \m{f}_s \\
        \end{bmatrix} $
        }
    \end{equation}

    which is equal to:
    \begin{equation}
        \scalebox{0.95}{
        $ \begin{bmatrix}
            \m{K}_{uu} & \m{K}_{um} + \m{K}_{us} \m{\closure{T}} \\
            (\m{K}_{um} + \m{K}_{us} \m{\closure{T}})^T &
            \m{K}_{mm} + \m{\closure{T}}^T \m{K}_{sm} +
            \m{K}_{sm}^T \m{\closure{T}} + \m{\closure{T}}^T \m{K}_{ss} \m{\closure{T}} \\
        \end{bmatrix}
        \begin{bmatrix}
            \m{u}_u \\
            \m{u}_m \\
        \end{bmatrix}
        = \begin{bmatrix}
            \m{f}_u \\
            \m{f}_m + \m{\closure{T}}^T \m{f}_s \\
        \end{bmatrix} $
        }
    \end{equation}

    considering that $ \m{f}_s = \m{0} $ the equations reduce to:
    \begin{equation}
        \begin{bmatrix}
            \m{K}_{uu} & \m{\hat{K}}_{um} \\
            \m{\hat{K}}_{um}^T & \m{\hat{K}}_{mm} \\
        \end{bmatrix}
        \begin{bmatrix}
            \m{u}_u \\
            \m{u}_m \\
        \end{bmatrix}
        = \begin{bmatrix}
            \m{f}_u \\
            \m{f}_m \\
        \end{bmatrix}
    \end{equation}

    where:
    \begin{eqarray}
        \m{\hat{K}}_{um} &= \m{K}_{um} + \m{K}_{us} \m{\closure{T}} \\
        \m{\hat{K}}_{mm} &= \m{K}_{mm} + \m{\closure{T}}^T \m{K}_{sm}
                          + \m{K}_{sm}^T \m{\closure{T}}
                          + \m{\closure{T}}^T \m{K}_{ss} \m{\closure{T}}
    \end{eqarray}

\end{qbox}


\begin{bbox}

    Then if \textbf{master-slave} relations are written as (\textit{general case}):
    \begin{equation}
        \m{u}_s = \m{\closure{T}} \m{u}_m + \m{g}
    \end{equation}

    Inserting into the partitioned master stiffness equations:
    \begin{equation}
        \begin{bmatrix}
            \m{K}_{uu} & \m{\hat{K}}_{um} \\
            \m{\hat{K}}_{um}^T & \m{\hat{K}}_{mm}
        \end{bmatrix}
        \begin{bmatrix}
            \m{u}_u \\
            \m{u}_m
        \end{bmatrix}
        = \begin{bmatrix}
            \m{f}_u - \m{K}_{us} \m{g} \\
            \m{f}_m - \m{K}_{ms} \m{g}
        \end{bmatrix}
    \end{equation}

    where:
    \begin{eqarray}
        \m{\hat{K}}_{um} &= \m{K}_{um} + \m{K}_{us} \m{\closure{T}} \\
        \m{\hat{K}}_{mm} &= \m{K}_{mm} + \m{\closure{T}}^T \m{K}_{ms}^T
                            + \m{K}_{ms} \m{\closure{T}}
                            + \m{\closure{T}}^T \m{K}_{ss} \m{\closure{T}}
    \end{eqarray}

    or?
    \begin{eqarray}
        \m{\hat{K}}_{um} &= \m{K}_{um} \m{\closure{T}} \\
        \m{\hat{K}}_{mm} &= \m{\closure{T}}^T \m{K}_{mm} \m{\closure{T}} \\
    \end{eqarray}

\end{bbox}

\end{document}
