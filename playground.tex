\documentclass[10pt,b5paper,titlepage]{book}

\usepackage[utf8]{inputenc}
\usepackage{amsmath}
\usepackage{amsfonts}
\usepackage{amssymb}
\usepackage{mathtools}
%\usepackage{xcolor}
\usepackage{color}
\usepackage{graphicx}
\usepackage[table]{xcolor}

% python - needs pip install Pygments
\usepackage{minted}
\usemintedstyle{native}

\usepackage{hyperref}

\author{Jan Tomek}
\title{\bf Playground}

% python code format
\setminted[python]{breaklines, framesep=2mm, fontsize=\footnotesize, numbersep=5pt}
\newminted[python]{python}{linenos=true,
                           frame=lines,
                           baselinestretch=1.2,
                           mathescape,
                           xleftmargin=0cm,
                           framesep=2mm,
                           fontsize=\footnotesize}

\setlength{\parindent}{0ex}
\setlength{\parskip}{1em plus 0.1em minus 0.2em}
\renewcommand{\labelitemi}{$\bullet$}
\renewcommand{\labelitemii}{$\bullet$}
\renewcommand{\labelitemiii}{$\bullet$}
\renewcommand{\labelitemiv}{$\bullet$}

%Commands definitions
\newcommand\setbackgroundcolour{\pagecolor[rgb]{0.15,0.15,0.15}}
\newcommand\settextcolour{\color[rgb]{0.9,0.9,0.9}}
\newcommand\invertbackgroundtext{\setbackgroundcolour\settextcolour}

\newcommand*{\Scale}[2][4]{\scalebox{#1}{$#2$}}%
\newcommand*{\Resize}[2]{\resizebox{#1}{!}{$#2$}}%

% Lagrangian symbol
\newcommand{\lagr}{\mathop{\mathcal{L}}}
\DeclareMathOperator{\Lagr}{\mathcal{L}}

% matrix notation
\newcommand{\m}{\mathbf}
% matrix notation greek letters
\newcommand{\M}{\pmb}

% null symbol
\newcommand{\Null}{\text{\O}}

% atan2 math symbol
\DeclareMathOperator{\atantwo}{atan2}
\DeclareMathOperator{\arctantwo}{arctan2}

\newcommand{\closure}[2][3]{%
{}\mkern#1mu\overline{\mkern-#1mu#2}}

% maximum number of entries in matrix
\setcounter{MaxMatrixCols}{20}

% equal by definition
\newcommand*\eqd{\stackrel{\triangle}{=}}

% {name}[number of arguments][1st default value] etc..
% the argument value is then inserted at #argument_number
\newenvironment{bbox}[1][0.96]
{
    \begin{center}
        \begin{tabular}{|p{#1\textwidth}|}
            \hline\\
}
{
            \\\\\hline
        \end{tabular}
    \end{center}
}

\newenvironment{bboxtitle}[1][0.96]
{
    \begin{center}
        #1\\[1ex]
        \begin{tabular}{|p{#1\textwidth}|}
            \hline\\
}
{
            \\\\\hline
        \end{tabular}
    \end{center}
}

\newcount\myloopcounter
\newcommand{\repeatit}[3][10]{%
    \myloopcounter1% initialize the loop counter
    \loop\ifnum\myloopcounter < #1
    #2#3%
    \advance\myloopcounter by 1%
    \repeat% start again
    #2%
}

\newenvironment{qbox}
{
%\centering{\huge{?}}
\begin{center}
    \repeatit[42]{?}{\ }
\end{center}
%\hrule
}
{
%\hrule
\begin{center}
    \repeatit[42]{?`}{\ }
\end{center}
}

\newenvironment{eqarray}
{
    \begin{eqnarray}
        \begin{aligned}
}
{
        \end{aligned}
    \end{eqnarray}
}

%Command execution.
%If this line is commented, then the appearance remains as usual.
\invertbackgroundtext

\begin{document}

\maketitle

\chapter{RBE2}

\begin{bbox}[0.95]
    The rigid freebody relations between \textbf{slave} and \textbf{master} node are:
    \begin{eqarray}
        \m{T}_s &= \m{T}_m + \m{R}_s \times \overline{\m{x}} \\
        \m{R}_s &= \m{R}_m
    \end{eqarray}

    Where $ \overline{\m{x}} $ is a vector from \textbf{slave} to \textbf{master}:
    \begin{equation}
        \overline{\m{x}} =
        \begin{bmatrix}
            \overline{x} \\
            \overline{y} \\
            \overline{z}
        \end{bmatrix} =
        \begin{bmatrix}
            x_s - x_m \\
            y_s - y_m \\
            z_s - z_m
        \end{bmatrix}
    \end{equation}

    Then \textit{vector multiplication} (cross product) is defined as:
    \begin{eqarray}
        \m{R}_s \times \m{x}_s
        &= \m{R}_m \times \m{x} = & \\
        &= \begin{bmatrix}
             \varphi_m \\
             \psi_m \\
             \theta_m
           \end{bmatrix} \times \begin{bmatrix}
             x_s - x_m \\
             y_s - y_m \\
             z_s - z_m
           \end{bmatrix}
        &= \begin{bmatrix}
             \varphi_m \\
             \psi_m \\
             \theta_m
           \end{bmatrix} \times \begin{bmatrix}
             \overline{x} \\
             \overline{y} \\
             \overline{z}
           \end{bmatrix} = \\
        &= \begin{bmatrix}
             \psi_m (z_s - z_m ) - \theta_m (y_s - y_m) \\
             \theta_m (x_s - x_m) - \varphi_m (z_s - z_m) \\
             \varphi_m (y_s - y_m) - \psi_m (x_s - x_m)
           \end{bmatrix}
        &= \begin{bmatrix}
            \psi_m \overline{z} - \theta_m \overline{y} \\
            \theta_m \overline{x} - \varphi_m \overline{z} \\
            \varphi_m \overline{y} - \psi_m \overline{x}
        \end{bmatrix} = \\
        &= \begin{bmatrix}
            \phantom{-}0 & \phantom{-}\overline{z} & -\overline{y} \\
            -\overline{z} & \phantom{-}0 & \phantom{-}\overline{x} \\
            \phantom{-}\overline{y} & -\overline{x} & \phantom{-}0
        \end{bmatrix}
        \begin{bmatrix}
            \varphi_m \\
            \psi_m \\
            \theta_m
        \end{bmatrix}
    \end{eqarray}
\end{bbox}

\begin{bbox}[0.95]
    The \textbf{slave} equations can be written as:
    \begin{equation}
        \begin{bmatrix}
            u_s \\
            v_s \\
            w_s \\
            \varphi_s \\
            \psi_s \\
            \theta_s
        \end{bmatrix}
        = \begin{bmatrix}
            1 & 0 & 0 & 0 & \phantom{-(}z_s - z_m & -(y_s - y_m) \\
            0 & 1 & 0 & -(z_s - z_m) & 0 & \phantom{-(}x_s - x_m \\
            0 & 0 & 1 & \phantom{-(}y_s - y_m & -(x_s - x_m) & 0 \\
            0 & 0 & 0 & 1 & 0 & 0 \\
            0 & 0 & 0 & 0 & 1 & 0 \\
            0 & 0 & 0 & 0 & 0 & 1 \\
        \end{bmatrix}
        \begin{bmatrix}
            u_m \\
            v_m \\
            w_m \\
            \varphi_m \\
            \psi_m \\
            \theta_m
        \end{bmatrix}
    \end{equation}

    or:
    \begin{equation}
        \begin{bmatrix}
            u_s \\
            v_s \\
            w_s \\
            \varphi_s \\
            \psi_s \\
            \theta_s
        \end{bmatrix}
        = \begin{bmatrix}
            1 & 0 & 0 & \phantom{-}0 & \phantom{-}\overline{z} & -\overline{y} \\
            0 & 1 & 0 & -\overline{z} & \phantom{-}0 & \phantom{-}\overline{x} \\
            0 & 0 & 1 & \phantom{-}\overline{y} & -\overline{x} & \phantom{-}0 \\
            0 & 0 & 0 & \phantom{-}1 & \phantom{-}0 & \phantom{-}0 \\
            0 & 0 & 0 & \phantom{-}0 & \phantom{-}1 & \phantom{-}0 \\
            0 & 0 & 0 & \phantom{-}0 & \phantom{-}0 & \phantom{-}1 \\
        \end{bmatrix}
        \begin{bmatrix}
            u_m \\
            v_m \\
            w_m \\
            \varphi_m \\
            \psi_m \\
            \theta_m
        \end{bmatrix}
    \end{equation}

    \begin{qbox}
        Finally the force and moment conditions have to be imposed:
        \begin{eqarray}
            \m{F}_m + \sum_{i=1}^n \m{F}_{s,i} &= \Null \\
            \m{M}_m + \sum_{i=1}^n \m{M}_{s,i} &= \Null \\
        \end{eqarray}

        which is equal to:
        \begin{eqarray}
            k_{m,u} u_m + \sum_{i=1}^n k_{s,u,i} u_{s,i} &= \Null \\
            k_{m,v} v_m + \sum_{i=1}^n k_{s,v,i} v_{s,i} &= \Null \\
            k_{m,w} w_m + \sum_{i=1}^n k_{s,w,i} w_{s,i} &= \Null \\
            k_{m,\varphi} \varphi_m + \sum_{i=1}^n
            \left(
                k_{s,\varphi,i} \varphi_{s,i}
                + k_{s,v,i} v_{s,i} \overline{z}_{i}
                + k_{s,w,i} w_{s,i} \overline{y}_{i}
        \right) &= \Null \\
            k_{m,\psi} \psi_m + \sum_{i=1}^n
            \left(
                k_{s,\psi,i} \psi_{s,i}
                + k_{s,u,i} u_{s,i} \overline{z}_{i}
                + k_{s,w,i} w_{s,i} \overline{x}_{i}
            \right) &= \Null \\
            k_{m,\theta} \theta_m + \sum_{i=1}^n
            \left(
                k_{s,\theta,i} \theta_{s,i}
                + k_{s,u,i} u_{s,i} \overline{y}_{i}
                + k_{s,v,i} v_{s,i} \overline{x}_{i}
            \right) &= \Null \\
        \end{eqarray}
    \end{qbox}

\end{bbox}

\section{RBE2 Coefficients Example:}

Master Node:
\begin{equation}
    \m{X}_m =
    \begin{bmatrix}
        1 & 1 & 1
    \end{bmatrix}^T
\end{equation}

Slave Nodes:
\begin{eqarray}
    \m{X}_{s,1} =
    \begin{bmatrix}
        0 & 0 & 0
    \end{bmatrix}^T \\
    \m{X}_{s,1} =
    \begin{bmatrix}
        2 & 0 & 0
    \end{bmatrix}^T \\
    \m{X}_{s,1} =
    \begin{bmatrix}
        2 & 2 & 0
    \end{bmatrix}^T \\
    \m{X}_{s,1} =
    \begin{bmatrix}
        0 & 2 & 0
    \end{bmatrix}^T \\
\end{eqarray}

Then Slave Node 1 equations are:
\begin{eqarray}
    \m{x}_1 &=
    \begin{bmatrix}
        x_{s,1} - x_m & y_{s,1} - y_m & z_{s,1} - z_m
    \end{bmatrix}^T =
    \begin{bmatrix}
        -1 & -1 & -1
    \end{bmatrix}^T
\end{eqarray}

Other Slave Nodes:
\begin{eqarray}
    \m{x}_2 &=
    \begin{bmatrix}
        2 - 1 & 0 - 1 & 0 - 1
    \end{bmatrix}^T =
    \begin{bmatrix}
        \phantom{-}1 & -1 & -1
    \end{bmatrix}^T \\
    \m{x}_3 &=
    \begin{bmatrix}
        2 - 1 & 2 - 1 & 0 - 1
    \end{bmatrix}^T =
    \begin{bmatrix}
        \phantom{-}1 & \phantom{-}1 & -1
    \end{bmatrix}^T \\
    \m{x}_4 &=
    \begin{bmatrix}
        0 - 1 & 2 - 1 & 0 - 1
    \end{bmatrix}^T =
    \begin{bmatrix}
        -1 & \phantom{-}1 & -1
    \end{bmatrix}^T
\end{eqarray}

Therefore, after filling in the coefficients:
\renewcommand\arraystretch{1.6}
\begin{eqarray}
    \begin{bmatrix}
        u_{s,1} \\
        v_{s,1} \\
        w_{s,1} \\
        \varphi_{s,1} \\
        \psi_{s,1} \\
        \theta_{s,1}
    \end{bmatrix}
    &= \begin{bmatrix}
        1 & 0 & 0 & \phantom{-}0 & -1 & \phantom{-}1 \\
        0 & 1 & 0 & \phantom{-}1 & \phantom{-}0 & -1 \\
        0 & 0 & 1 & -1 & \phantom{-}1 & \phantom{-}0 \\
        0 & 0 & 0 & \phantom{-}1 & \phantom{-}0 & \phantom{-}0 \\
        0 & 0 & 0 & \phantom{-}0 & \phantom{-}1 & \phantom{-}0 \\
        0 & 0 & 0 & \phantom{-}0 & \phantom{-}0 & \phantom{-}1 \\
    \end{bmatrix}
    \begin{bmatrix}
        u_m \\
        v_m \\
        w_m \\
        \varphi_m \\
        \psi_m \\
        \theta_m
    \end{bmatrix} \\
    \begin{bmatrix}
        u_{s,2} \\
        v_{s,2} \\
        w_{s,2} \\
        \varphi_{s,2} \\
        \psi_{s,2} \\
        \theta_{s,2}
    \end{bmatrix}
    &= \begin{bmatrix}
        1 & 0 & 0 & \phantom{-}0 & -1 & \phantom{-}1 \\
        0 & 1 & 0 & \phantom{-}1 & \phantom{-}0 & \phantom{-}1 \\
        0 & 0 & 1 & -1 & -1 & \phantom{-}0 \\
        0 & 0 & 0 & \phantom{-}1 & \phantom{-}0 & \phantom{-}0 \\
        0 & 0 & 0 & \phantom{-}0 & \phantom{-}1 & \phantom{-}0 \\
        0 & 0 & 0 & \phantom{-}0 & \phantom{-}0 & \phantom{-}1 \\
    \end{bmatrix}
    \begin{bmatrix}
        u_m \\
        v_m \\
        w_m \\
        \varphi_m \\
        \psi_m \\
        \theta_m
    \end{bmatrix} \\
    \begin{bmatrix}
        u_{s,3} \\
        v_{s,3} \\
        w_{s,3} \\
        \varphi_{s,3} \\
        \psi_{s,3} \\
        \theta_{s,3}
    \end{bmatrix}
    &= \begin{bmatrix}
        1 & 0 & 0 & \phantom{-}0 & -1 & -1 \\
        0 & 1 & 0 & \phantom{-}1 & \phantom{-}0 & \phantom{-}1 \\
        0 & 0 & 1 & \phantom{-}1 & -1 & \phantom{-}0 \\
        0 & 0 & 0 & \phantom{-}1 & \phantom{-}0 & \phantom{-}0 \\
        0 & 0 & 0 & \phantom{-}0 & \phantom{-}1 & \phantom{-}0 \\
        0 & 0 & 0 & \phantom{-}0 & \phantom{-}0 & \phantom{-}1 \\
    \end{bmatrix}
    \begin{bmatrix}
        u_m \\
        v_m \\
        w_m \\
        \varphi_m \\
        \psi_m \\
        \theta_m
    \end{bmatrix} \\
    \begin{bmatrix}
        u_{s,4} \\
        v_{s,4} \\
        w_{s,4} \\
        \varphi_{s,4} \\
        \psi_{s,4} \\
        \theta_{s,4}
    \end{bmatrix}
    &= \begin{bmatrix}
        1 & 0 & 0 & \phantom{-}0 & -1 & -1 \\
        0 & 1 & 0 & \phantom{-}1 & \phantom{-}0 & -1 \\
        0 & 0 & 1 & \phantom{-}1 & \phantom{-}1 & \phantom{-}0 \\
        0 & 0 & 0 & \phantom{-}1 & \phantom{-}0 & \phantom{-}0 \\
        0 & 0 & 0 & \phantom{-}0 & \phantom{-}1 & \phantom{-}0 \\
        0 & 0 & 0 & \phantom{-}0 & \phantom{-}0 & \phantom{-}1 \\
    \end{bmatrix}
    \begin{bmatrix}
        u_m \\
        v_m \\
        w_m \\
        \varphi_m \\
        \psi_m \\
        \theta_m
    \end{bmatrix}
\end{eqarray}
\renewcommand\arraystretch{1}

When all put together:
\begin{equation}
    \begin{bmatrix}
        u_{s,1} \\
        v_{s,1} \\
        w_{s,1} \\
        \varphi_{s,1} \\
        \psi_{s,1} \\
        \theta_{s,1} \\
        u_{s,2} \\
        v_{s,2} \\
        w_{s,2} \\
        \varphi_{s,2} \\
        \psi_{s,2} \\
        \theta_{s,2} \\
        u_{s,3} \\
        v_{s,3} \\
        w_{s,3} \\
        \varphi_{s,3} \\
        \psi_{s,3} \\
        \theta_{s,3} \\
        u_{s,4} \\
        v_{s,4} \\
        w_{s,4} \\
        \varphi_{s,4} \\
        \psi_{s,4} \\
        \theta_{s,4}
    \end{bmatrix}
    = \begin{bmatrix}
        1 & 0 & 0 & \phantom{-}0 & -1 & \phantom{-}1 \\
        0 & 1 & 0 & \phantom{-}1 & \phantom{-}0 & -1 \\
        0 & 0 & 1 & -1 & \phantom{-}1 & \phantom{-}0 \\
        0 & 0 & 0 & \phantom{-}1 & \phantom{-}0 & \phantom{-}0 \\
        0 & 0 & 0 & \phantom{-}0 & \phantom{-}1 & \phantom{-}0 \\
        0 & 0 & 0 & \phantom{-}0 & \phantom{-}0 & \phantom{-}1 \\
        1 & 0 & 0 & \phantom{-}0 & -1 & \phantom{-}1 \\
        0 & 1 & 0 & \phantom{-}1 & \phantom{-}0 & \phantom{-}1 \\
        0 & 0 & 1 & -1 & -1 & \phantom{-}0 \\
        0 & 0 & 0 & \phantom{-}1 & \phantom{-}0 & \phantom{-}0 \\
        0 & 0 & 0 & \phantom{-}0 & \phantom{-}1 & \phantom{-}0 \\
        0 & 0 & 0 & \phantom{-}0 & \phantom{-}0 & \phantom{-}1 \\
        1 & 0 & 0 & \phantom{-}0 & -1 & -1 \\
        0 & 1 & 0 & \phantom{-}1 & \phantom{-}0 & \phantom{-}1 \\
        0 & 0 & 1 & \phantom{-}1 & -1 & \phantom{-}0 \\
        0 & 0 & 0 & \phantom{-}1 & \phantom{-}0 & \phantom{-}0 \\
        0 & 0 & 0 & \phantom{-}0 & \phantom{-}1 & \phantom{-}0 \\
        0 & 0 & 0 & \phantom{-}0 & \phantom{-}0 & \phantom{-}1 \\
        1 & 0 & 0 & \phantom{-}0 & -1 & -1 \\
        0 & 1 & 0 & \phantom{-}1 & \phantom{-}0 & -1 \\
        0 & 0 & 1 & \phantom{-}1 & \phantom{-}1 & \phantom{-}0 \\
        0 & 0 & 0 & \phantom{-}1 & \phantom{-}0 & \phantom{-}0 \\
        0 & 0 & 0 & \phantom{-}0 & \phantom{-}1 & \phantom{-}0 \\
        0 & 0 & 0 & \phantom{-}0 & \phantom{-}0 & \phantom{-}1 \\
    \end{bmatrix}
    \begin{bmatrix}
        u_m \\
        v_m \\
        w_m \\
        \varphi_m \\
        \psi_m \\
        \theta_m
    \end{bmatrix}
\end{equation}

It can be read as:
\begin{eqarray}
    u_{s,1} &= u_m -\psi_m + \theta_m \\
    v_{s,1} &= v_m \varphi_m - \theta_m \\
    \vdots \\
    \theta_{s,4} &= \theta_m
\end{eqarray}

Or:
\begin{equation}\label{rbe2-coefficients}
    \m{u}_s = \m{\closure{T}} \m{u}_m
\end{equation}


\section{Editing Master Sitffness Matrix}

This equation should be then expanded to all DOFs, where in $ \m{T} $ the
slave DOFs columns are missing and the DOFs that do not take part in RBEs have $ 1 $ on
the diagonal. The resulting  matrix should have the same number of rows as
the master stiffness matrix.

Then the master stiffness equation is modified:
\begin{eqarray}\label{rbe2-master-slave-modified}
    \m{\hat{K}} &= \m{T}^T \m{K} \m{T} \\
    \m{\hat{f}} &= \m{T}^T \m{f} \\
    \m{\hat{K}} \m{\hat{u}} &= \m{\hat{f}} \\
\end{eqarray}

If we'd like to make the procedure more general, partition the master stiffness
equations such that:
\begin{equation}
    \begin{bmatrix}
        \m{K}_{uu} & \m{K}_{um} & \m{K}_{us} \\
        \m{K}_{um}^T & \m{K}_{mm} & \m{K}_{ms} \\
        \m{K}_{us}^T & \m{K}_{ms}^T & \m{K}_{ss}
    \end{bmatrix}
    \begin{bmatrix}
        \m{u}_u \\
        \m{u}_m \\
        \m{u}_s
    \end{bmatrix}
    = \begin{bmatrix}
        \m{f}_u \\
        \m{f}_m \\
        \m{f}_s
    \end{bmatrix}
\end{equation}

\begin{qbox}
    Based on \eqref{rbe2-master-slave-modified} we could probably partition
    the matrices as:
    \begin{equation}\label{rbe2-partition-T-u-m-s}
        \begin{matrix}
            \begin{matrix}
                \begin{matrix}
                    \ 
                \end{matrix} \\
                \left. \begin{matrix}
                    u \\
                    m \\
                    s \\
                \end{matrix} \right|
            \end{matrix}
            & \begin{matrix}
                \begin{matrix}
                    u & m \\
                    \hline
                \end{matrix} \\
                \begin{bmatrix}
                    \m{I} & \m{0} \\
                    \m{0} & \m{I} \\
                    \m{0} & \m{\closure{T}} \\
                \end{bmatrix} \\
            \end{matrix}
        \end{matrix}
    \end{equation}

    If we expand \eqref{rbe2-coefficients} and partition the matrices
    based on \eqref{rbe2-partition-T-u-m-s} into independent or
    \textbf{unconstrained}, \textbf{master}
    and \textbf{slave} nodes, then the equation:
    \begin{equation}
        \m{u} = \m{T} \m{\hat{u}}
    \end{equation}

    becomes:
    \begin{equation}
        \begin{bmatrix}
            \m{u}_u  \\
            \m{u}_m \\
            \m{u}_s \\
        \end{bmatrix}
        = \begin{bmatrix}
            \m{I} & \m{0} \\
            \m{0} & \m{I} \\
            \m{0} & \m{\closure{T}} \\
        \end{bmatrix}
        \begin{bmatrix}
            \m{u}_u \\
            \m{u}_m \\
        \end{bmatrix}
    \end{equation}


    Finally the equation \eqref{rbe2-master-slave-modified} can be rewritten as:
    \begin{equation}
        \begin{bmatrix}
            \m{I} & \m{0} & \m{0} \\
            \m{0} & \m{I} & \m{\closure{T}}^T \\
        \end{bmatrix}
        \begin{bmatrix}
            \m{K}_{uu} & \m{K}_{um} & \m{K}_{us} \\
            \m{K}_{mu} & \m{K}_{mm} & \m{K}_{ms} \\
            \m{K}_{su} & \m{K}_{sm} & \m{K}_{ss} \\
        \end{bmatrix}
        \begin{bmatrix}
            \m{I} & \m{0} \\
            \m{0} & \m{I} \\
            \m{0} & \m{\closure{T}} \\
        \end{bmatrix}
        \begin{bmatrix}
            \m{u}_u \\
            \m{u}_m \\
        \end{bmatrix}
        = \begin{bmatrix}
            \m{I} & \m{0} & \m{0} \\
            \m{0} & \m{I} & \m{\closure{T}}^T \\
        \end{bmatrix}
        \begin{bmatrix}
            \m{f}_u \\
            \m{f}_m \\
            \m{f}_s \\
        \end{bmatrix}
    \end{equation}

    Expanding and multiplying:
    \begin{equation}
        \scalebox{0.92}{
        $ \begin{bmatrix}
            \m{K}_{uu} & \m{K}_{um} & \m{K}_{us} \\
            \m{K}_{mu} + \m{\closure{T}}^T \m{K}_{su} &
            \m{K}_{mm} + \m{\closure{T}}^T \m{K}_{sm} &
            \m{K}_{ms} + \m{\closure{T}}^T \m{K}_{ss} \\
        \end{bmatrix}
        \begin{bmatrix}
            \m{I} & \m{0} \\
            \m{0} & \m{I} \\
            \m{0} & \m{\closure{T}} \\
        \end{bmatrix}
        \begin{bmatrix}
            \m{u}_u \\
            \m{u}_m \\
        \end{bmatrix}
        = \begin{bmatrix}
            \m{f}_u \\
            \m{f}_m + \m{\closure{T}}^T \m{f}_s \\
        \end{bmatrix} $
        }
    \end{equation}

    \begin{equation}
        \scalebox{0.95}{
        $ \begin{bmatrix}
            \m{K}_{uu} & \m{K}_{um} + \m{K}_{us} \m{\closure{T}} \\
            \m{K}_{mu} + \m{\closure{T}}^T \m{K}_{su} &
            \m{K}_{mm} + \m{\closure{T}}^T \m{K}_{sm} +
            \m{K}_{ms} \m{\closure{T}} + \m{\closure{T}}^T \m{K}_{ss} \m{\closure{T}} \\
        \end{bmatrix}
        \begin{bmatrix}
            \m{u}_u \\
            \m{u}_m \\
        \end{bmatrix}
        = \begin{bmatrix}
            \m{f}_u \\
            \m{f}_m + \m{\closure{T}}^T \m{f}_s \\
        \end{bmatrix} $
        }
    \end{equation}

    which is equal to:
    \begin{equation}
        \scalebox{0.95}{
        $ \begin{bmatrix}
            \m{K}_{uu} & \m{K}_{um} + \m{K}_{us} \m{\closure{T}} \\
            (\m{K}_{um} + \m{K}_{us} \m{\closure{T}})^T &
            \m{K}_{mm} + \m{\closure{T}}^T \m{K}_{ms}^T +
            \m{K}_{ms} \m{\closure{T}} + \m{\closure{T}}^T \m{K}_{ss} \m{\closure{T}} \\
        \end{bmatrix}
        \begin{bmatrix}
            \m{u}_u \\
            \m{u}_m \\
        \end{bmatrix}
        = \begin{bmatrix}
            \m{f}_u \\
            \m{f}_m + \m{\closure{T}}^T \m{f}_s \\
        \end{bmatrix} $
        }
    \end{equation}

    considering that $ \m{f}_s = \m{0} $ the equations reduce to:
    \begin{equation}
        \begin{bmatrix}
            \m{K}_{uu} & \m{\hat{K}}_{um} \\
            \m{\hat{K}}_{um}^T & \m{\hat{K}}_{mm} \\
        \end{bmatrix}
        \begin{bmatrix}
            \m{u}_u \\
            \m{u}_m \\
        \end{bmatrix}
        = \begin{bmatrix}
            \m{f}_u \\
            \m{f}_m \\
        \end{bmatrix}
    \end{equation}

    where:
    \begin{eqarray}
        \m{\hat{K}}_{um} &= \m{K}_{um} + \m{K}_{us} \m{\closure{T}} \\
        \m{\hat{K}}_{mm} &= \m{K}_{mm} + \m{\closure{T}}^T \m{K}_{ms}^T
                          + \m{K}_{ms} \m{\closure{T}}
                          + \m{\closure{T}}^T \m{K}_{ss} \m{\closure{T}} \\
                         &= \m{K}_{mm} + \m{K}_{ms} \m{\closure{T}}
                          + (\m{K}_{ms} \m{\closure{T}})^T
                          + \m{\closure{T}}^T \m{K}_{ss} \m{\closure{T}}
    \end{eqarray}

\end{qbox}


\begin{bbox}

    Then if \textbf{master-slave} relations are written as (\textit{general case}):
    \begin{equation}
        \m{u}_s = \m{\closure{T}} \m{u}_m + \m{g}
    \end{equation}

    Inserting into the partitioned master stiffness equations:
    \begin{equation}
        \begin{bmatrix}
            \m{K}_{uu} & \m{\hat{K}}_{um} \\
            \m{\hat{K}}_{um}^T & \m{\hat{K}}_{mm}
        \end{bmatrix}
        \begin{bmatrix}
            \m{u}_u \\
            \m{u}_m
        \end{bmatrix}
        = \begin{bmatrix}
            \m{f}_u - \m{K}_{us} \m{g} \\
            \m{f}_m - \m{K}_{ms} \m{g}
        \end{bmatrix}
    \end{equation}

    where:
    \begin{eqarray}
        \m{\hat{K}}_{um} &= \m{K}_{um} + \m{K}_{us} \m{\closure{T}} \\
        \m{\hat{K}}_{mm} &= \m{K}_{mm} + \m{\closure{T}}^T \m{K}_{ms}^T
                            + \m{K}_{ms} \m{\closure{T}}
                            + \m{\closure{T}}^T \m{K}_{ss} \m{\closure{T}}
    \end{eqarray}

\end{bbox}


\newpage
\section{RBE2 Example 2D with ROD Elements}

\subsection{Recap}

\subsubsection{Stiffness Matrix}

ROD Stiffness Matrix:
\begin{equation}
    \m{K}_l^e = \frac{E^e A^e}{L^e} \begin{bmatrix}
        \phantom{-}1 & 0 & -1 & 0 \\
        \phantom{-}0 & 0 & \phantom{-}0 & 0 \\
        -1 & 0 & \phantom{-}1 & 0 \\
        \phantom{-}0 & 0 & \phantom{-}0 & 0 \\
    \end{bmatrix}
\end{equation}

ROD Transformation Matrix:
\begin{eqarray}
    \m{u}_l &= \prescript{e}{}{\m{T}} \m{u}_g \\
    \prescript{e}{}{\m{T}}^T \m{u}_l &=
        \prescript{e}{}{\m{T}}^T \prescript{e}{}{\m{T}} \m{u}_g \\
    \prescript{e}{}{\m{T}}^T \m{u}_l &= \m{I} \m{u}_g \\
    \prescript{e}{}{\m{T}}^T \m{u}_l &= \m{u}_g \\
    \prescript{e}{}{\m{T}} &=
    \begin{bmatrix}
        \phantom{-}c & s & \phantom{-}0 & 0 \\
        -s & c & \phantom{-}0 & 0 \\
        \phantom{-}0 & 0 & \phantom{-}c & s \\
        \phantom{-}0 & 0 & -s & c \\
    \end{bmatrix} \\
\end{eqarray}

where:
\begin{eqarray}
    s &= \sin \alpha^e &= \frac{y_2 - y_1}{L^e} \\
    c &= \cos \alpha^e &= \frac{x_2 - x_1}{L^e} \\
\end{eqarray}

Then the \textbf{global master stiffness matrix} is:

\begin{equation}
    \prescript{e}{}{\m{T}}^T \m{K}_l^e \prescript{e}{}{\m{T}} \m{u}_g
    = \prescript{e}{}{\m{T}}^T \m{f}_l
\end{equation}

Or:
\begin{equation}
    \m{K}_g \m{u}_g = \m{f}_g
\end{equation}


\subsubsection{MPC Application}

Afterwards the \textbf{MPCs} are applied:
\begin{equation}
    \m{u}_s = \m{\closure{T}} \m{u}_m
\end{equation}

This, when applied to the whole model:
\begin{equation}
    \begin{bmatrix}
        \m{u}_u  \\
        \m{u}_m \\
        \m{u}_s \\
    \end{bmatrix}
    = \begin{bmatrix}
        \m{I} & \m{0} \\
        \m{0} & \m{I} \\
        \m{0} & \m{\closure{T}} \\
    \end{bmatrix}
    \begin{bmatrix}
        \m{u}_u \\
        \m{u}_m \\
    \end{bmatrix}
\end{equation}

with \textbf{transformation matrix} $ \m{T} $ being:
\begin{equation}
    \m{T} = \begin{bmatrix}
        \m{I} & \m{0} \\
        \m{0} & \m{I} \\
        \m{0} & \m{\closure{T}} \\
    \end{bmatrix}
\end{equation}

Applying the transformation matrix to get \textbf{modified master equation system}:
\begin{eqarray}
    \m{T}^T \m{K}_g \m{T} \m{\hat{u}}_g &= \m{T}^T \m{f}_g \\
    \m{\hat{K}}_g \m{\hat{u}}_g &= \m{\hat{f}}_g
\end{eqarray}

Where $ \m{\hat{u}}_g = \begin{bmatrix} \m{u}_{g,u} & \m{u}_{g,m} \end{bmatrix}^T $.

From now on, the subscript $ _g $ (\textit{as global}) is implied.


\subsubsection{Rotated Nodal Basis}

When a node or an SPC is defined as a \textbf{rotated}/\textbf{skewed} or in
a different type of Coordinate system (e.g. cylindrical) a transformation matrix
needs to be applied to such nodes.

\textbf{Cartesian Coordinate system} definition by angle (2D):
\begin{equation}
    \m{u}_l = \m{T} \m{u}_g
\end{equation}

Unpacked:
\begin{equation}
    \begin{bmatrix}
        u_l \\
        w_l \\
        \psi_l \\
    \end{bmatrix}
    = \begin{bmatrix}
        \phantom{-}\cos \alpha & \sin \alpha & 0 \\
        -\sin \alpha & \cos \alpha & 0 \\
        0 & 0 & 1 \\
    \end{bmatrix}
    \begin{bmatrix}
        u_g \\
        w_g \\
        \psi_g \\
    \end{bmatrix}
\end{equation}

\textbf{Cartesian Coordinate system} definition by vectors (2D):

when a \textbf{CSYS} is defined by an \textbf{origin} and \textbf{two vectors}
$ \m{x} = \begin{pmatrix} x_1 & x_2 & x_3 \end{pmatrix} $
and $ \m{y} = \begin{pmatrix} y_1 & y_2 & y_3 \end{pmatrix} $,
then for a Cartesian CSYS it suffices to:
\begin{eqarray}
    \m{\closure{x}} &= \frac{\m{x}}{||\m{x}||} \\
    \m{\closure{y}} &= \frac{\m{y}}{||\m{y}||} \\
    \m{\closure{z}} &= \m{\closure{x}} \times \m{\closure{y}} \\
\end{eqarray}

The transformation relation:
\begin{equation}
    \m{u}_l = \m{T} \m{u}_g
\end{equation}

when unpacked:
\begin{equation}
    \begin{bmatrix}
        u_l \\
        w_l \\
        \psi_l \\
    \end{bmatrix}
    = \begin{bmatrix}
        \m{x} \\
        \m{y} \\
        \m{z} \\
    \end{bmatrix}
    \begin{bmatrix}
        u_g \\
        w_g \\
        \psi_g \\
    \end{bmatrix}
\end{equation}

For a \textbf{2D case} just \textbf{one vector} is enough, the
$ \m{\closure{z}} = \begin{pmatrix} 0 & 0 & 1 \end{pmatrix} $ and following relations
apply:
\begin{eqarray}
    \m{\closure{x}} &= \m{\closure{y}} \times \m{\closure{z}} \\
    \m{\closure{y}} &= \m{\closure{z}} \times \m{\closure{x}} \\
    \m{\closure{z}} &= \m{\closure{x}} \times \m{\closure{y}} \\
\end{eqarray}

This means that a \textbf{CSYS} defined by an \textbf{origin} point
$ \m{O} = \begin{pmatrix} x_O & y_O & z_O \end{pmatrix} $,
point laying on \textbf{x axis}
$ \m{P}_x = \begin{pmatrix} x_{P_x} & y_{P_x} & z_{P_x} \end{pmatrix} $ and a
point laying in \textbf{xy plane}
$ \m{P}_{xy} = \begin{pmatrix} x_{P_{xy}} & y_{P_{xy}} & z_{P_{xy}} \end{pmatrix} $
define a CSYS followingly:
\begin{eqarray}
    \m{x} &= \begin{pmatrix} x_{P_x} - x_O & y_{P_x} - y_O & z_{P_x} - z_O \end{pmatrix} \\
    \m{\closure{x}} &= \frac{\m{x}}{||\m{x}||} \\
    \m{\hat{y}} &= \begin{pmatrix} x_{P_{xy}} - x_O & y_{P_{xy}} - y_O & z_{P_{xy}} - z_O \end{pmatrix} \\
    \m{\closure{\hat{y}}} &= \frac{\m{\hat{y}}}{||\m{\hat{y}}||} \\
    \m{\closure{z}} &= \m{\closure{x}} \times \m{\closure{\hat{y}}} \\
    \m{\closure{y}} &= \m{\closure{z}} \times \m{\closure{x}} \\
\end{eqarray}


\textbf{Cylindrical Coordinate system} definition (2D):

\begin{qbox}
    The basic relation between a point in \textbf{cartesian} and \textbf{cylindrical}
    CSYS is (when the cartesian CSYS has the same origin as the cylindrical one
    and both CSYS $ z $ axes overlay):
    \begin{eqarray}
        x &= r \cos \varphi \\
        y &= r \sin \varphi \\
        z &= z \\
    \end{eqarray}

     in one direction and:
     \begin{eqarray}
         r &= \sqrt{x^2 + y^2} \\
         \varphi &= \left\{ \begin{array}{cc}
                    indeterminate & if\ x = 0\ and\ y = 0 \\
                    \arcsin \frac{y}{r} & if\ x \ge 0 \\
                    -\arcsin \frac{y}{r} + \pi & if\ x < 0\ and\ y \ge 0 \\
                    -\arcsin \frac{y}{r} + \pi & if\ x < 0\ and\ y < 0 \\
                 \end{array} \right.
     \end{eqarray}

     in the other. One can also use the $ \arctan $ function to compute $ \varphi $:
     \begin{equation}
         \varphi = \left\{ \begin{array}{cc}
                 indeterminate & if\ x = 0\ and\ y = 0 \\
                 \frac{\pi}{2}\frac{y}{|y|} & if\ x = 0\ and\ y \ne 0 \\
                 \arctan \frac{y}{x} & if\ x > 0 \\
                 \arctan \frac{y}{x} + \pi & if\ x < 0\ and\ y \ge 0 \\
                 \arctan \frac{y}{x} - \pi & if\ x < 0\ and\ y < 0 \\
             \end{array} \right.
     \end{equation}

     This is also called the $ \atantwo (y, x) $ function!

     To transform node coordinates from \textbf{cartesian} to \textbf{cylindrical},
     three transforms are in order:

     \begin{enumerate}
         \item translate the coordinates so that \textbf{origin} of the new
             CSYS conforms to the \textbf{origin} of the \textbf{cylidrical}
             CSYS.

             \begin{equation}
                 \m{x}_1 = \m{T}_T \m{x}_0
             \end{equation}

             \begin{equation}
                 \begin{bmatrix}
                     x_1 \\
                     y_1 \\
                     z_1 \\
                     1 \\
                 \end{bmatrix}
                 = \begin{bmatrix}
                     1 & 0 & 0 & -x_O^c \\
                     0 & 1 & 0 & -y_O^c \\
                     0 & 0 & 1 & -z_O^c \\
                     0 & 0 & 0 & \phantom{-}1 \\
                 \end{bmatrix}
                 \begin{bmatrix}
                     x_0 \\
                     y_0 \\
                     z_0 \\
                     1 \\
                 \end{bmatrix}
             \end{equation}

             where $ \m{x}_O^c $ denotes the \textbf{cylindrical} CSYS origin
             and $ \m{T}_T $ denotes the translation matrix.

        \item \textbf{rotate} the node to a new \textbf{cartesian} CSYS so that
            the new CSYS \textbf{x-axis} conforms to the cylindrical CSYS
            \textbf{r-axis} and the new CSYS \textbf{z-axis} conforms to the
            cylindrical \textbf{z-axis}.

            \begin{equation}
                \m{x}_2 = \m{T}_R \m{x}_1
            \end{equation}

            When the cylindrical CSYS is defined by three points - \textbf{origin}
             $ \m{x}_O^c = \begin{bmatrix} x_O^C & y_O^c & z_O^c \end{bmatrix}^T $,
             point on \textbf{r-axis} $ \m{x}_{R}^c $
            and a point in \textbf{rz-plane} $ \m{x}_{RZ}^c $, then:

            \begin{equation}
                \vec{\m{r}} = \frac{\m{x}_R^c - \m{x}_O^c}{||\m{x}_R^c - \m{x}_O^c||}
            \end{equation}

            \begin{equation}
                \hat{\m{z}} = \frac{\m{x}_{RZ}^c - \m{x}_O^c}{||\m{x}_{RZ}^c - \m{x}_O^c||}
            \end{equation}

            \begin{equation}
                \vec{\m{y}} = \hat{\m{z}} \times \vec{\m{r}}
            \end{equation}

            \begin{equation}
                \vec{\m{z}} = \vec{\m{r}} \times \vec{\m{y}}
            \end{equation}

            then the transformation can be written:
            \begin{equation}
                \begin{bmatrix}
                    x_2 \\
                    y_2 \\
                    z_2 \\
                    1 \\
                \end{bmatrix}
                = \begin{bmatrix}
                    \vec{\m{r}}^T & 0 \\
                    \vec{\m{y}}^T & 0 \\
                    \vec{\m{z}}^T & 0 \\
                    \vec{\m{0}}^T & 1 \\
                \end{bmatrix}
                \begin{bmatrix}
                    x_1 \\
                    y_1 \\
                    z_1 \\
                    1 \\
                \end{bmatrix}
            \end{equation}

             where $ \vec{\m{0}} = \begin{bmatrix} 0 & 0 & 0 \end{bmatrix}^T $

             Combining $ \m{T}_R $ and $ \m{T}_T $ together:
             \begin{equation}
                 \m{T} = \m{T}_R \m{T}_T
             \end{equation}

             \begin{equation}
                 \m{T} = \begin{bmatrix}
                     r_1 & r_2 & r_3 & 0 \\
                     y_1 & y_2 & y_3 & 0 \\
                     z_1 & z_2 & z_3 & 0 \\
                 \end{bmatrix}
                 \begin{bmatrix}
                     1 & 0 & 0 & -x_O^c \\
                     0 & 1 & 0 & -y_O^c \\
                     0 & 0 & 1 & -z_O^c \\
                     0 & 0 & 0 & \phantom{-}1 \\
                 \end{bmatrix}
             \end{equation}

             \begin{equation}
                 \m{T} = \begin{bmatrix}
                     r_1 & r_2 & r_3 & -r_1 x_O^c -r_2 y_O^c -r_3 z_O^c  \\
                     y_1 & y_2 & y_3 & -y_1 x_O^c -y_2 y_O^c -y_3 z_O^c  \\
                     z_1 & z_2 & z_3 & -z_1 x_O^c -z_2 y_O^c -z_3 z_O^c  \\
                 \end{bmatrix}
             \end{equation}


        \item transform the coordinates to the \textbf{cylindrical} CSYS.
            \begin{equation}
                \begin{bmatrix}
                    r \\
                    \varphi \\
                    z \\
                \end{bmatrix}
                = \begin{bmatrix}
                    \sqrt{x_2^2 + y_2^2} \\
                    \atantwo \left(y_2, x_2\right) \\
                    z_2 \\
                \end{bmatrix}
            \end{equation}

    \end{enumerate}

\end{qbox}



\subsubsection{SPC Application}

Then to solve the \textbf{master stiffness equations} for prescribed \textbf{BCs}
partitioned to \textbf{unconstrained} and \textbf{master} DOFs such as:
\begin{equation}
    \begin{bmatrix}
        \m{K}_{uu} & \m{\hat{K}}_{um} \\
        \m{\hat{K}}_{um}^T & \m{\hat{K}}_{mm} \\
    \end{bmatrix}
    \begin{bmatrix}
        \m{u}_u \\
        \m{u}_m \\
    \end{bmatrix}
    = \begin{bmatrix}
        \m{f}_{u} \\
        \m{f}_m \\
    \end{bmatrix}
\end{equation}

where:
\begin{eqarray}
    \m{\hat{K}}_{um} &= \m{K}_{um} + \m{K}_{us} \m{\closure{T}} \\
    \m{\hat{K}}_{mm} &= \m{K}_{mm} + \m{K}_{ms} \m{\closure{T}}
                      + (\m{K}_{ms} \m{\closure{T}})^T
                      + \m{\closure{T}}^T \m{K}_{ss} \m{\closure{T}}
\end{eqarray}

Any of these \textbf{DOFs} can be constrained, so this partitioning scheme looses
all purposes. Therefore the matrix is repartitioned again into
\textbf{independent} and \textbf{constrained} partitions:
\begin{equation}
    \begin{bmatrix}
        \m{K}_{ii} & \m{K}_{ic} \\
        \m{K}_{ic}^T & \m{K}_{cc} \\
    \end{bmatrix}
    \begin{bmatrix}
        \m{u}_i \\
        \m{u}_c \\
    \end{bmatrix}
    = \begin{bmatrix}
        \m{f}_{i} \\
        \m{f}_c \\
    \end{bmatrix}
\end{equation}


\subsubsection{Solving Linear Equations}

Where the \textbf{unknowns} are $ \m{u}_i $, $ \m{f}_c $:
\begin{equation}
    \begin{bmatrix}
        \m{K}_{ii} & \m{K}_{ic} \\
        \m{K}_{ic}^T & \m{K}_{cc} \\
    \end{bmatrix}
    \begin{bmatrix}
        \boxed{\m{u}_i} \\
        \m{u}_c \\
    \end{bmatrix}
    = \begin{bmatrix}
        \m{f}_i \\
        \boxed{\m{f}_c} \\
    \end{bmatrix}
\end{equation}

Unpacking the equations:
\begin{eqarray}
    \m{K}_{ii} \boxed{\m{u}_i} &+ \m{K}_{ic} \m{u}_c &= \m{f}_i \\
    \m{K}_{ic}^T \boxed{\m{u}_i} &+ \m{K}_{cc} \m{u}_c &= \boxed{\m{f}_c} \\
\end{eqarray}

Then first solve for the \textbf{unknown displacements}:
\begin{eqarray}
    \m{K}_{ii} \boxed{\m{u}_i} &= \m{f}_u - \m{K}_{ic} \m{u}_c \\
    \boxed{\m{u}_i} &= \m{K}_{ii}^{-1} \left( \m{f}_i - \m{K}_{ic} \m{u}_c \right) \\
\end{eqarray}

Lastly solve for the \textbf{unknown reaction forces}:
\begin{equation}
    \boxed{\m{f}_c} = \m{K}_{ii} \m{u}_i + \m{K}_{ic} \m{u}_c
\end{equation}


\subsubsection{Recovering Full Displacement Vector}

The full displacement vector is then recovered:
\begin{eqarray}
    \m{T} &= \begin{bmatrix}
        \m{I} & \m{0} \\
        \m{0} & \m{I} \\
        \m{0} & \m{\closure{T}} \\
    \end{bmatrix} \\
    \m{\hat{u}} &= \begin{bmatrix}
        \m{u}_i \\
        \m{u}_c \\
    \end{bmatrix}
    = \begin{bmatrix}
        \m{u}_u \\
        \m{u}_m \\
    \end{bmatrix} \\
    \m{u}_g &= \m{T} \m{\hat{u}} \\
    \begin{bmatrix}
        \m{u}_u \\
        \m{u}_m \\
        \m{u}_s \\
    \end{bmatrix}
    &= \begin{bmatrix}
        \m{I} & \m{0} \\
        \m{0} & \m{I} \\
        \m{0} & \m{\closure{T}} \\
    \end{bmatrix}
    \begin{bmatrix}
        \m{u}_u \\
        \m{u}_m \\
    \end{bmatrix}
\end{eqarray}




\end{document}
