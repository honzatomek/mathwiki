\newpage
\chapter{Model reduction}

\section{Guyan reduction}
\textbf{Guyan reduction}, also known as \textbf{static condensation},
is a \textit{dimensionality reduction} method which reduces the number
of \textbf{DOFs} by ignoring the interial terms of the equilibrium equations
and expressing the unloaded \textbf{DOFs} in terms of the loaded \textbf{DOFs}.

The static equilibrium equation:

\begin{eqarray}
    \begin{bmatrix}
        \m{K}_{mm} & \m{K}_{ms} \\
        \m{K}_{sm} & \m{K}_{ss}
    \end{bmatrix}
    \begin{bmatrix}
        \m{u}_m \\
        \m{u}_s
    \end{bmatrix} &=
    \begin{bmatrix}
        \m{f}_m \\
        \m{f}_s
    \end{bmatrix} \\
    \m{K}_{mm} \m{u}_m + \m{K}_{ms} \m{u}_s &= \m{f}_m \\
    \m{K}_{sm} \m{u}_m + \m{K}_{ss} \m{u}_s &= \m{f}_s
\end{eqarray}

Where $ m $ denotes the \textbf{master} DOFs and $ s $ the \textbf{slave} DOFs.
Stating that $ s $ section of the problem DOFs is unloaded, we can
write that $ \m{f}_s = \m{0} $. Then the slave DOFs are expressed
by the following equation:

\begin{equation}
    \m{K}_{sm} \m{u}_m + \m{K}_{ss} \m{u}_s = \m{0}
\end{equation}

Solving the above equation in terms of the independent (master) DOFs leads to
the following dependency relations:

\begin{equation}
    \m{u}_s = -\m{K}_{ss}^{-1} \m{K}_{sm} \m{u}_m
\end{equation}

Substituting the dependency relations to the upper (master) partition of the static
equilibrium problem condenses away the slave DOFs, leading to the following reduced
system of linear equations:

\begin{equation}
    \left(\m{K}_{mm} - \m{K}_{ms} \m{K}_{ss}^{-1} \m{K}_{sm} \right)
    \m{u}_m = \m{f}_m
\end{equation}

The above system of linear equations is equivalent to the original problem, but expressed
in terms of \textbf{master} DOFs alone. Thus, the \textbf{Guyan} reduction method
results in a reduced system by condensing away the \textbf{slave} DOFs.

The \textbf{Guyan reduction} can be also expressed as a \textit{change of basis} which
produces a low-dimensional representation of the original space, represented by the
\textbf{master} DOFs. The linear transformation that maps the reduced space onto
the full space is expressed as:

\begin{equation}
    \begin{bmatrix}
        \m{u}_m \\
        \m{u}_s
    \end{bmatrix} =
    \begin{bmatrix}
        \m{I} \\
        -\m{K}_{ss}^{-1} \m{K}_{sm}
    \end{bmatrix}
    \m{u}_m = \m{T}_G \m{u}_m
\end{equation}

where $ \m{T}_G $ represents the \textbf{Guyan} reduction \textit{transformation matrix}.
Thus, the reduced problem is represented as:

\begin{equation}
    \m{K}_G \m{u}_m = \m{f}_m
\end{equation}

In the above equation, $ \m{K}_G $ represents the reduced system of linear equations
that's obtained by applying the \textbf{Guyan reduction} transformation on the
full system, which is expressed as:

\begin{equation}
    \m{K}_G = \m{T}_G^T \m{K} \m{T}_G
\end{equation}

where $ \m{T}_G $ has the dimension of $ n_{DOF} \times n_m $.


