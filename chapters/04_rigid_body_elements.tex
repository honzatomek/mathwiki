\newpage
\chapter{Rigid Body Elements}

\section{RBE2}

\begin{bbox}[0.95]
    The rigid freebody relations between \textbf{slave} and \textbf{master} node are:
    \begin{eqarray}
        \m{T}_s &= \m{T}_m + \m{R}_s \times \overline{\m{x}} \\
        \m{R}_s &= \m{R}_m
    \end{eqarray}

    Where $ \overline{\m{x}} $ is a vector from \textbf{slave} to \textbf{master}:
    \begin{equation}
        \overline{\m{x}} =
        \begin{bmatrix}
            \overline{x} \\
            \overline{y} \\
            \overline{z}
        \end{bmatrix} =
        \begin{bmatrix}
            x_s - x_m \\
            y_s - y_m \\
            z_s - z_m
        \end{bmatrix}
    \end{equation}

    Then \textit{vector multiplication} (cross product) is defined as:
    \begin{eqarray}
        \m{R}_s \times \m{x}_s
        &= \m{R}_m \times \m{x} = & \\
        &= \begin{bmatrix}
             \varphi_m \\
             \psi_m \\
             \theta_m
           \end{bmatrix} \times \begin{bmatrix}
             x_s - x_m \\
             y_s - y_m \\
             z_s - z_m
           \end{bmatrix}
        &= \begin{bmatrix}
             \varphi_m \\
             \psi_m \\
             \theta_m
           \end{bmatrix} \times \begin{bmatrix}
             \overline{x} \\
             \overline{y} \\
             \overline{z}
           \end{bmatrix} = \\
        &= \begin{bmatrix}
             \psi_m (z_s - z_m ) - \theta_m (y_s - y_m) \\
             \theta_m (x_s - x_m) - \varphi_m (z_s - z_m) \\
             \varphi_m (y_s - y_m) - \psi_m (x_s - x_m)
           \end{bmatrix}
        &= \begin{bmatrix}
            \psi_m \overline{z} - \theta_m \overline{y} \\
            \theta_m \overline{x} - \varphi_m \overline{z} \\
            \varphi_m \overline{y} - \psi_m \overline{x}
        \end{bmatrix} = \\
        &= \begin{bmatrix}
            \phantom{-}0 & \phantom{-}\overline{z} & -\overline{y} \\
            -\overline{z} & \phantom{-}0 & \phantom{-}\overline{x} \\
            \phantom{-}\overline{y} & -\overline{x} & \phantom{-}0
        \end{bmatrix}
        \begin{bmatrix}
            \varphi_m \\
            \psi_m \\
            \theta_m
        \end{bmatrix}
    \end{eqarray}
\end{bbox}

\begin{bbox}[0.95]
    The \textbf{slave} equations can be written as:
    \begin{equation}
        \begin{bmatrix}
            u_s \\
            v_s \\
            w_s \\
            \varphi_s \\
            \psi_s \\
            \theta_s
        \end{bmatrix}
        = \begin{bmatrix}
            1 & 0 & 0 & 0 & \phantom{-(}z_s - z_m & -(y_s - y_m) \\
            0 & 1 & 0 & -(z_s - z_m) & 0 & \phantom{-(}x_s - x_m \\
            0 & 0 & 1 & \phantom{-(}y_s - y_m & -(x_s - x_m) & 0 \\
            0 & 0 & 0 & 1 & 0 & 0 \\
            0 & 0 & 0 & 0 & 1 & 0 \\
            0 & 0 & 0 & 0 & 0 & 1 \\
        \end{bmatrix}
        \begin{bmatrix}
            u_m \\
            v_m \\
            w_m \\
            \varphi_m \\
            \psi_m \\
            \theta_m
        \end{bmatrix}
    \end{equation}

    or:
    \begin{equation}
        \begin{bmatrix}
            u_s \\
            v_s \\
            w_s \\
            \varphi_s \\
            \psi_s \\
            \theta_s
        \end{bmatrix}
        = \begin{bmatrix}
            1 & 0 & 0 & \phantom{-}0 & \phantom{-}\overline{z} & -\overline{y} \\
            0 & 1 & 0 & -\overline{z} & \phantom{-}0 & \phantom{-}\overline{x} \\
            0 & 0 & 1 & \phantom{-}\overline{y} & -\overline{x} & \phantom{-}0 \\
            0 & 0 & 0 & \phantom{-}1 & \phantom{-}0 & \phantom{-}0 \\
            0 & 0 & 0 & \phantom{-}0 & \phantom{-}1 & \phantom{-}0 \\
            0 & 0 & 0 & \phantom{-}0 & \phantom{-}0 & \phantom{-}1 \\
        \end{bmatrix}
        \begin{bmatrix}
            u_m \\
            v_m \\
            w_m \\
            \varphi_m \\
            \psi_m \\
            \theta_m
        \end{bmatrix}
    \end{equation}
\end{bbox}

\newpage
\subsection{RBE2 Coefficients Example:}

Master Node:
\begin{equation}
    \m{X}_m =
    \begin{bmatrix}
        1 & 1 & 1
    \end{bmatrix}^T
\end{equation}

Slave Nodes:
\begin{eqarray}
    \m{X}_{s,1} =
    \begin{bmatrix}
        0 & 0 & 0
    \end{bmatrix}^T \\
    \m{X}_{s,1} =
    \begin{bmatrix}
        2 & 0 & 0
    \end{bmatrix}^T \\
    \m{X}_{s,1} =
    \begin{bmatrix}
        2 & 2 & 0
    \end{bmatrix}^T \\
    \m{X}_{s,1} =
    \begin{bmatrix}
        0 & 2 & 0
    \end{bmatrix}^T \\
\end{eqarray}

Then Slave Node 1 equations are:
\begin{eqarray}
    \m{x}_1 &=
    \begin{bmatrix}
        x_{s,1} - x_m & y_{s,1} - y_m & z_{s,1} - z_m
    \end{bmatrix}^T =
    \begin{bmatrix}
        -1 & -1 & -1
    \end{bmatrix}^T
\end{eqarray}

Other Slave Nodes:
\begin{eqarray}
    \m{x}_2 &=
    \begin{bmatrix}
        2 - 1 & 0 - 1 & 0 - 1
    \end{bmatrix}^T =
    \begin{bmatrix}
        \phantom{-}1 & -1 & -1
    \end{bmatrix}^T \\
    \m{x}_3 &=
    \begin{bmatrix}
        2 - 1 & 2 - 1 & 0 - 1
    \end{bmatrix}^T =
    \begin{bmatrix}
        \phantom{-}1 & \phantom{-}1 & -1
    \end{bmatrix}^T \\
    \m{x}_4 &=
    \begin{bmatrix}
        0 - 1 & 2 - 1 & 0 - 1
    \end{bmatrix}^T =
    \begin{bmatrix}
        -1 & \phantom{-}1 & -1
    \end{bmatrix}^T
\end{eqarray}

Therefore, after filling in the coefficients:
\renewcommand\arraystretch{1.6}
\begin{eqarray}
    \begin{bmatrix}
        u_{s,1} \\
        v_{s,1} \\
        w_{s,1} \\
        \varphi_{s,1} \\
        \psi_{s,1} \\
        \theta_{s,1}
    \end{bmatrix}
    &= \begin{bmatrix}
        1 & 0 & 0 & \phantom{-}0 & -1 & \phantom{-}1 \\
        0 & 1 & 0 & \phantom{-}1 & \phantom{-}0 & -1 \\
        0 & 0 & 1 & -1 & \phantom{-}1 & \phantom{-}0 \\
        0 & 0 & 0 & \phantom{-}1 & \phantom{-}0 & \phantom{-}0 \\
        0 & 0 & 0 & \phantom{-}0 & \phantom{-}1 & \phantom{-}0 \\
        0 & 0 & 0 & \phantom{-}0 & \phantom{-}0 & \phantom{-}1 \\
    \end{bmatrix}
    \begin{bmatrix}
        u_m \\
        v_m \\
        w_m \\
        \varphi_m \\
        \psi_m \\
        \theta_m
    \end{bmatrix} \\
    \begin{bmatrix}
        u_{s,2} \\
        v_{s,2} \\
        w_{s,2} \\
        \varphi_{s,2} \\
        \psi_{s,2} \\
        \theta_{s,2}
    \end{bmatrix}
    &= \begin{bmatrix}
        1 & 0 & 0 & \phantom{-}0 & -1 & \phantom{-}1 \\
        0 & 1 & 0 & \phantom{-}1 & \phantom{-}0 & \phantom{-}1 \\
        0 & 0 & 1 & -1 & -1 & \phantom{-}0 \\
        0 & 0 & 0 & \phantom{-}1 & \phantom{-}0 & \phantom{-}0 \\
        0 & 0 & 0 & \phantom{-}0 & \phantom{-}1 & \phantom{-}0 \\
        0 & 0 & 0 & \phantom{-}0 & \phantom{-}0 & \phantom{-}1 \\
    \end{bmatrix}
    \begin{bmatrix}
        u_m \\
        v_m \\
        w_m \\
        \varphi_m \\
        \psi_m \\
        \theta_m
    \end{bmatrix} \\
    \begin{bmatrix}
        u_{s,3} \\
        v_{s,3} \\
        w_{s,3} \\
        \varphi_{s,3} \\
        \psi_{s,3} \\
        \theta_{s,3}
    \end{bmatrix}
    &= \begin{bmatrix}
        1 & 0 & 0 & \phantom{-}0 & -1 & -1 \\
        0 & 1 & 0 & \phantom{-}1 & \phantom{-}0 & \phantom{-}1 \\
        0 & 0 & 1 & \phantom{-}1 & -1 & \phantom{-}0 \\
        0 & 0 & 0 & \phantom{-}1 & \phantom{-}0 & \phantom{-}0 \\
        0 & 0 & 0 & \phantom{-}0 & \phantom{-}1 & \phantom{-}0 \\
        0 & 0 & 0 & \phantom{-}0 & \phantom{-}0 & \phantom{-}1 \\
    \end{bmatrix}
    \begin{bmatrix}
        u_m \\
        v_m \\
        w_m \\
        \varphi_m \\
        \psi_m \\
        \theta_m
    \end{bmatrix} \\
    \begin{bmatrix}
        u_{s,4} \\
        v_{s,4} \\
        w_{s,4} \\
        \varphi_{s,4} \\
        \psi_{s,4} \\
        \theta_{s,4}
    \end{bmatrix}
    &= \begin{bmatrix}
        1 & 0 & 0 & \phantom{-}0 & -1 & -1 \\
        0 & 1 & 0 & \phantom{-}1 & \phantom{-}0 & -1 \\
        0 & 0 & 1 & \phantom{-}1 & \phantom{-}1 & \phantom{-}0 \\
        0 & 0 & 0 & \phantom{-}1 & \phantom{-}0 & \phantom{-}0 \\
        0 & 0 & 0 & \phantom{-}0 & \phantom{-}1 & \phantom{-}0 \\
        0 & 0 & 0 & \phantom{-}0 & \phantom{-}0 & \phantom{-}1 \\
    \end{bmatrix}
    \begin{bmatrix}
        u_m \\
        v_m \\
        w_m \\
        \varphi_m \\
        \psi_m \\
        \theta_m
    \end{bmatrix}
\end{eqarray}
\renewcommand\arraystretch{1}

When all put together:
\begin{equation}
    \begin{bmatrix}
        u_{s,1} \\
        v_{s,1} \\
        w_{s,1} \\
        \varphi_{s,1} \\
        \psi_{s,1} \\
        \theta_{s,1} \\
        u_{s,2} \\
        v_{s,2} \\
        w_{s,2} \\
        \varphi_{s,2} \\
        \psi_{s,2} \\
        \theta_{s,2} \\
        u_{s,3} \\
        v_{s,3} \\
        w_{s,3} \\
        \varphi_{s,3} \\
        \psi_{s,3} \\
        \theta_{s,3} \\
        u_{s,4} \\
        v_{s,4} \\
        w_{s,4} \\
        \varphi_{s,4} \\
        \psi_{s,4} \\
        \theta_{s,4}
    \end{bmatrix}
    = \begin{bmatrix}
        1 & 0 & 0 & \phantom{-}0 & -1 & \phantom{-}1 \\
        0 & 1 & 0 & \phantom{-}1 & \phantom{-}0 & -1 \\
        0 & 0 & 1 & -1 & \phantom{-}1 & \phantom{-}0 \\
        0 & 0 & 0 & \phantom{-}1 & \phantom{-}0 & \phantom{-}0 \\
        0 & 0 & 0 & \phantom{-}0 & \phantom{-}1 & \phantom{-}0 \\
        0 & 0 & 0 & \phantom{-}0 & \phantom{-}0 & \phantom{-}1 \\
        1 & 0 & 0 & \phantom{-}0 & -1 & \phantom{-}1 \\
        0 & 1 & 0 & \phantom{-}1 & \phantom{-}0 & \phantom{-}1 \\
        0 & 0 & 1 & -1 & -1 & \phantom{-}0 \\
        0 & 0 & 0 & \phantom{-}1 & \phantom{-}0 & \phantom{-}0 \\
        0 & 0 & 0 & \phantom{-}0 & \phantom{-}1 & \phantom{-}0 \\
        0 & 0 & 0 & \phantom{-}0 & \phantom{-}0 & \phantom{-}1 \\
        1 & 0 & 0 & \phantom{-}0 & -1 & -1 \\
        0 & 1 & 0 & \phantom{-}1 & \phantom{-}0 & \phantom{-}1 \\
        0 & 0 & 1 & \phantom{-}1 & -1 & \phantom{-}0 \\
        0 & 0 & 0 & \phantom{-}1 & \phantom{-}0 & \phantom{-}0 \\
        0 & 0 & 0 & \phantom{-}0 & \phantom{-}1 & \phantom{-}0 \\
        0 & 0 & 0 & \phantom{-}0 & \phantom{-}0 & \phantom{-}1 \\
        1 & 0 & 0 & \phantom{-}0 & -1 & -1 \\
        0 & 1 & 0 & \phantom{-}1 & \phantom{-}0 & -1 \\
        0 & 0 & 1 & \phantom{-}1 & \phantom{-}1 & \phantom{-}0 \\
        0 & 0 & 0 & \phantom{-}1 & \phantom{-}0 & \phantom{-}0 \\
        0 & 0 & 0 & \phantom{-}0 & \phantom{-}1 & \phantom{-}0 \\
        0 & 0 & 0 & \phantom{-}0 & \phantom{-}0 & \phantom{-}1 \\
    \end{bmatrix}
    \begin{bmatrix}
        u_m \\
        v_m \\
        w_m \\
        \varphi_m \\
        \psi_m \\
        \theta_m
    \end{bmatrix}
\end{equation}

It can be read as:
\begin{eqarray}
    u_{s,1} &= u_m -\psi_m + \theta_m \\
    v_{s,1} &= v_m \varphi_m - \theta_m \\
    \vdots \\
    \theta_{s,4} &= \theta_m
\end{eqarray}

Or:
\begin{equation}\label{rbe2-coefficients}
    \m{u}_s = \m{\closure{T}} \m{u}_m
\end{equation}


\subsection{Editing the Master Stiffness Matrix}

This equation should be then expanded to all DOFs, where in $ \m{T} $ the
slave DOFs columns are missing and the DOFs that do not take part in RBEs have $ 1 $ on
the diagonal. The resulting  matrix should have the same number of rows as
the master stiffness matrix.

Then the master stiffness equation is modified:
\begin{eqarray}\label{rbe2-master-slave-modified}
    \m{\hat{K}} &= \m{T}^T \m{K} \m{T} \\
    \m{\hat{f}} &= \m{T}^T \m{f} \\
    \m{\hat{K}} \m{\hat{u}} &= \m{\hat{f}} \\
\end{eqarray}

If we'd like to make the procedure more general, partition the master stiffness
equations such that:
\begin{equation}
    \begin{bmatrix}
        \m{K}_{uu} & \m{K}_{um} & \m{K}_{us} \\
        \m{K}_{um}^T & \m{K}_{mm} & \m{K}_{ms} \\
        \m{K}_{us}^T & \m{K}_{ms}^T & \m{K}_{ss}
    \end{bmatrix}
    \begin{bmatrix}
        \m{u}_u \\
        \m{u}_m \\
        \m{u}_s
    \end{bmatrix}
    = \begin{bmatrix}
        \m{f}_u \\
        \m{f}_m \\
        \m{f}_s
    \end{bmatrix}
\end{equation}

Based on \eqref{rbe2-master-slave-modified} we could probably partition
the matrices as:
\begin{equation}\label{rbe2-partition-T-u-m-s}
    \begin{matrix}
        \begin{matrix}
            \begin{matrix}
                \ 
            \end{matrix} \\
            \left. \begin{matrix}
                u \\
                m \\
                s \\
            \end{matrix} \right|
        \end{matrix}
        & \begin{matrix}
            \begin{matrix}
                u & m \\
                \hline
            \end{matrix} \\
            \begin{bmatrix}
                \m{I} & \m{0} \\
                \m{0} & \m{I} \\
                \m{0} & \m{\closure{T}} \\
            \end{bmatrix} \\
        \end{matrix}
    \end{matrix}
\end{equation}

If we expand \eqref{rbe2-coefficients} and partition the matrices
based on \eqref{rbe2-partition-T-u-m-s} into independent or
\textbf{unconstrained}, \textbf{master}
and \textbf{slave} nodes, then the equation:
\begin{equation}
    \m{u} = \m{T} \m{\hat{u}}
\end{equation}

becomes:
\begin{equation}
    \begin{bmatrix}
        \m{u}_u  \\
        \m{u}_m \\
        \m{u}_s \\
    \end{bmatrix}
    = \begin{bmatrix}
        \m{I} & \m{0} \\
        \m{0} & \m{I} \\
        \m{0} & \m{\closure{T}} \\
    \end{bmatrix}
    \begin{bmatrix}
        \m{u}_u \\
        \m{u}_m \\
    \end{bmatrix}
\end{equation}


Finally the equation \eqref{rbe2-master-slave-modified} can be rewritten as:
\begin{equation}
    \begin{bmatrix}
        \m{I} & \m{0} & \m{0} \\
        \m{0} & \m{I} & \m{\closure{T}}^T \\
    \end{bmatrix}
    \begin{bmatrix}
        \m{K}_{uu} & \m{K}_{um} & \m{K}_{us} \\
        \m{K}_{mu} & \m{K}_{mm} & \m{K}_{ms} \\
        \m{K}_{su} & \m{K}_{sm} & \m{K}_{ss} \\
    \end{bmatrix}
    \begin{bmatrix}
        \m{I} & \m{0} \\
        \m{0} & \m{I} \\
        \m{0} & \m{\closure{T}} \\
    \end{bmatrix}
    \begin{bmatrix}
        \m{u}_u \\
        \m{u}_m \\
    \end{bmatrix}
    = \begin{bmatrix}
        \m{I} & \m{0} & \m{0} \\
        \m{0} & \m{I} & \m{\closure{T}}^T \\
    \end{bmatrix}
    \begin{bmatrix}
        \m{f}_u \\
        \m{f}_m \\
        \m{f}_s \\
    \end{bmatrix}
\end{equation}

Expanding and multiplying:
\begin{equation}
    \scalebox{0.92}{
    $ \begin{bmatrix}
        \m{K}_{uu} & \m{K}_{um} & \m{K}_{us} \\
        \m{K}_{mu} + \m{\closure{T}}^T \m{K}_{su} &
        \m{K}_{mm} + \m{\closure{T}}^T \m{K}_{sm} &
        \m{K}_{ms} + \m{\closure{T}}^T \m{K}_{ss} \\
    \end{bmatrix}
    \begin{bmatrix}
        \m{I} & \m{0} \\
        \m{0} & \m{I} \\
        \m{0} & \m{\closure{T}} \\
    \end{bmatrix}
    \begin{bmatrix}
        \m{u}_u \\
        \m{u}_m \\
    \end{bmatrix}
    = \begin{bmatrix}
        \m{f}_u \\
        \m{f}_m + \m{\closure{T}}^T \m{f}_s \\
    \end{bmatrix} $
    }
\end{equation}

\begin{equation}
    \scalebox{0.95}{
    $ \begin{bmatrix}
        \m{K}_{uu} & \m{K}_{um} + \m{K}_{us} \m{\closure{T}} \\
        \m{K}_{mu} + \m{\closure{T}}^T \m{K}_{su} &
        \m{K}_{mm} + \m{\closure{T}}^T \m{K}_{sm} +
        \m{K}_{ms} \m{\closure{T}} + \m{\closure{T}}^T \m{K}_{ss} \m{\closure{T}} \\
    \end{bmatrix}
    \begin{bmatrix}
        \m{u}_u \\
        \m{u}_m \\
    \end{bmatrix}
    = \begin{bmatrix}
        \m{f}_u \\
        \m{f}_m + \m{\closure{T}}^T \m{f}_s \\
    \end{bmatrix} $
    }
\end{equation}

which is equal to:
\begin{equation}
    \scalebox{0.95}{
    $ \begin{bmatrix}
        \m{K}_{uu} & \m{K}_{um} + \m{K}_{us} \m{\closure{T}} \\
        (\m{K}_{um} + \m{K}_{us} \m{\closure{T}})^T &
        \m{K}_{mm} + \m{\closure{T}}^T \m{K}_{ms}^T +
        \m{K}_{ms} \m{\closure{T}} + \m{\closure{T}}^T \m{K}_{ss} \m{\closure{T}} \\
    \end{bmatrix}
    \begin{bmatrix}
        \m{u}_u \\
        \m{u}_m \\
    \end{bmatrix}
    = \begin{bmatrix}
        \m{f}_u \\
        \m{f}_m + \m{\closure{T}}^T \m{f}_s \\
    \end{bmatrix} $
    }
\end{equation}

considering that $ \m{f}_s = \m{0} $ the equations reduce to:
\begin{equation}
    \begin{bmatrix}
        \m{K}_{uu} & \m{\hat{K}}_{um} \\
        \m{\hat{K}}_{um}^T & \m{\hat{K}}_{mm} \\
    \end{bmatrix}
    \begin{bmatrix}
        \m{u}_u \\
        \m{u}_m \\
    \end{bmatrix}
    = \begin{bmatrix}
        \m{f}_u \\
        \m{f}_m \\
    \end{bmatrix}
\end{equation}

where:
\begin{eqarray}
    \m{\hat{K}}_{um} &= \m{K}_{um} + \m{K}_{us} \m{\closure{T}} \\
    \m{\hat{K}}_{mm} &= \m{K}_{mm} + \m{\closure{T}}^T \m{K}_{ms}^T
                      + \m{K}_{ms} \m{\closure{T}}
                      + \m{\closure{T}}^T \m{K}_{ss} \m{\closure{T}} \\
                     &= \m{K}_{mm} + \m{K}_{ms} \m{\closure{T}}
                      + (\m{K}_{ms} \m{\closure{T}})^T
                      + \m{\closure{T}}^T \m{K}_{ss} \m{\closure{T}}
\end{eqarray}


\begin{bbox}

    Then if \textbf{master-slave} relations are written as (\textit{general case}):
    \begin{equation}
        \m{u}_s = \m{\closure{T}} \m{u}_m + \m{g}
    \end{equation}

    Inserting into the partitioned master stiffness equations:
    \begin{equation}
        \begin{bmatrix}
            \m{K}_{uu} & \m{\hat{K}}_{um} \\
            \m{\hat{K}}_{um}^T & \m{\hat{K}}_{mm}
        \end{bmatrix}
        \begin{bmatrix}
            \m{u}_u \\
            \m{u}_m
        \end{bmatrix}
        = \begin{bmatrix}
            \m{f}_u - \m{K}_{us} \m{g} \\
            \m{f}_m - \m{K}_{ms} \m{g}
        \end{bmatrix}
    \end{equation}

    where:
    \begin{eqarray}
        \m{\hat{K}}_{um} &= \m{K}_{um} + \m{K}_{us} \m{\closure{T}} \\
        \m{\hat{K}}_{mm} &= \m{K}_{mm} + \m{\closure{T}}^T \m{K}_{ms}^T
                            + \m{K}_{ms} \m{\closure{T}}
                            + \m{\closure{T}}^T \m{K}_{ss} \m{\closure{T}}
    \end{eqarray}

\end{bbox}


\subsubsection{MPC Application}

Afterwards the \textbf{MPCs} are applied:
\begin{equation}
    \m{u}_s = \m{\closure{T}} \m{u}_m
\end{equation}

This, when applied to the whole model:
\begin{equation}
    \begin{bmatrix}
        \m{u}_u  \\
        \m{u}_m \\
        \m{u}_s \\
    \end{bmatrix}
    = \begin{bmatrix}
        \m{I} & \m{0} \\
        \m{0} & \m{I} \\
        \m{0} & \m{\closure{T}} \\
    \end{bmatrix}
    \begin{bmatrix}
        \m{u}_u \\
        \m{u}_m \\
    \end{bmatrix}
\end{equation}

with \textbf{transformation matrix} $ \m{T} $ being:
\begin{equation}
    \m{T} = \begin{bmatrix}
        \m{I} & \m{0} \\
        \m{0} & \m{I} \\
        \m{0} & \m{\closure{T}} \\
    \end{bmatrix}
\end{equation}

Applying the transformation matrix to get \textbf{modified master equation system}:
\begin{eqarray}
    \m{T}^T \m{K}_g \m{T} \m{\hat{u}}_g &= \m{T}^T \m{f}_g \\
    \m{\hat{K}}_g \m{\hat{u}}_g &= \m{\hat{f}}_g
\end{eqarray}

Where $ \m{\hat{u}}_g = \begin{bmatrix} \m{u}_{g,u} & \m{u}_{g,m} \end{bmatrix}^T $.

From now on, the subscript $ _g $ (\textit{as global}) is implied.


\subsubsection{Rotated Nodal Basis}

When a node or an SPC is defined as a \textbf{rotated}/\textbf{skewed} or in
a different type of Coordinate system (e.g. cylindrical) a transformation matrix
needs to be applied to such nodes.

\textbf{Cartesian Coordinate system} definition by angle (2D):
\begin{equation}
    \m{u}_l = \m{T} \m{u}_g
\end{equation}


\subsubsection{SPC Application}

Then to solve the \textbf{master stiffness equations} for prescribed \textbf{BCs}
partitioned to \textbf{unconstrained} and \textbf{master} DOFs such as:
\begin{equation}
    \begin{bmatrix}
        \m{K}_{uu} & \m{\hat{K}}_{um} \\
        \m{\hat{K}}_{um}^T & \m{\hat{K}}_{mm} \\
    \end{bmatrix}
    \begin{bmatrix}
        \m{u}_u \\
        \m{u}_m \\
    \end{bmatrix}
    = \begin{bmatrix}
        \m{f}_{u} \\
        \m{f}_m \\
    \end{bmatrix}
\end{equation}

where:
\begin{eqarray}
    \m{\hat{K}}_{um} &= \m{K}_{um} + \m{K}_{us} \m{\closure{T}} \\
    \m{\hat{K}}_{mm} &= \m{K}_{mm} + \m{K}_{ms} \m{\closure{T}}
                      + (\m{K}_{ms} \m{\closure{T}})^T
                      + \m{\closure{T}}^T \m{K}_{ss} \m{\closure{T}}
\end{eqarray}

Any of these \textbf{DOFs} can be constrained, so this partitioning scheme looses
all purposes. Therefore the matrix is repartitioned again into
\textbf{independent} and \textbf{constrained} partitions:
\begin{equation}
    \begin{bmatrix}
        \m{K}_{ii} & \m{K}_{ic} \\
        \m{K}_{ic}^T & \m{K}_{cc} \\
    \end{bmatrix}
    \begin{bmatrix}
        \m{u}_i \\
        \m{u}_c \\
    \end{bmatrix}
    = \begin{bmatrix}
        \m{f}_{i} \\
        \m{f}_c \\
    \end{bmatrix}
\end{equation}


\subsubsection{Solving Linear Equations}

Where the \textbf{unknowns} are $ \m{u}_i $, $ \m{f}_c $:
\begin{equation}
    \begin{bmatrix}
        \m{K}_{ii} & \m{K}_{ic} \\
        \m{K}_{ic}^T & \m{K}_{cc} \\
    \end{bmatrix}
    \begin{bmatrix}
        \boxed{\m{u}_i} \\
        \m{u}_c \\
    \end{bmatrix}
    = \begin{bmatrix}
        \m{f}_i \\
        \boxed{\m{f}_c} \\
    \end{bmatrix}
\end{equation}

Unpacking the equations:
\begin{eqarray}
    \m{K}_{ii} \boxed{\m{u}_i} &+ \m{K}_{ic} \m{u}_c &= \m{f}_i \\
    \m{K}_{ic}^T \boxed{\m{u}_i} &+ \m{K}_{cc} \m{u}_c &= \boxed{\m{f}_c} \\
\end{eqarray}

Then first solve for the \textbf{unknown displacements}:
\begin{eqarray}
    \m{K}_{ii} \boxed{\m{u}_i} &= \m{f}_u - \m{K}_{ic} \m{u}_c \\
    \boxed{\m{u}_i} &= \m{K}_{ii}^{-1} \left( \m{f}_i - \m{K}_{ic} \m{u}_c \right) \\
\end{eqarray}

Lastly solve for the \textbf{unknown reaction forces}:
\begin{equation}
    \boxed{\m{f}_c} = \m{K}_{ii} \m{u}_i + \m{K}_{ic} \m{u}_c
\end{equation}


\subsubsection{Recovering Full Displacement Vector}

The full displacement vector is then recovered:
\begin{eqarray}
    \m{T} &= \begin{bmatrix}
        \m{I} & \m{0} \\
        \m{0} & \m{I} \\
        \m{0} & \m{\closure{T}} \\
    \end{bmatrix} \\
    \m{\hat{u}} &= \begin{bmatrix}
        \m{u}_i \\
        \m{u}_c \\
    \end{bmatrix}
    = \begin{bmatrix}
        \m{u}_u \\
        \m{u}_m \\
    \end{bmatrix} \\
    \m{u}_g &= \m{T} \m{\hat{u}} \\
    \begin{bmatrix}
        \m{u}_u \\
        \m{u}_m \\
        \m{u}_s \\
    \end{bmatrix}
    &= \begin{bmatrix}
        \m{I} & \m{0} \\
        \m{0} & \m{I} \\
        \m{0} & \m{\closure{T}} \\
    \end{bmatrix}
    \begin{bmatrix}
        \m{u}_u \\
        \m{u}_m \\
    \end{bmatrix}
\end{eqarray}



\newpage
\section{RBE3}

The first step is to calculate the \textbf{weighted COG} of the element, where
\begin{equation}
    \m{x}_{COG} = \frac{\sum w_i \m{x}_i}{\sum w_i}
\end{equation}

where $ w_i $ is a weighting factor of DOF $ i $ and
$ \m{x}_i $ is the position of the node related with DOF $ i $

Once the \textbf{COG} is found, the loading is transloated from the reference
point $ \m{x}_{app} $ where it is applied to the \textbf{COG} by performing a
equilibrium equivalency.

\begin{equation}
    \m{F}_{COG} = \m{F}_{app}
\end{equation}

\begin{equation}
    \m{M}_{COG} = \m{M}_{app} + \m{r} \times \m{F}_{app}
\end{equation}

Then the $ \m{F}_{COG} $ is reacted (shared) between all master nodes relative
to their weight.

\begin{equation}
    \m{F}_i = \m{F}_{COG} \frac{w_i}{\sum w_i}
\end{equation}

The moment $ \m{M}_{COG} $ is assumed to be reacted (shared) by the mater DOFs
according to the relative weight of each one of them and the relative position
with regards to the $ COG $. The reaction force that compensates the moment is
in the direction perpendicular to the vector joining the $ COG $ with the
position of the node where the master DOFs are:

\begin{equation}
    \m{P}_i = -w_i \frac{\m{M}_{COG} \times \m{r}_i }{\sum w_i . \m{r}_i . \m{r}_i}
\end{equation}

where $ \m{r}_i $ is the vector from the $ COG $ position to the $ i $-th node
$ \m{r}_i = \m{x}_i - \m{x}_{COG} $.

The total reaction force on the master DOFs is the sum $ \m{F}_i + \m{P}_i $ for
each master DOF.

\textbf{This force is not dependent on the stiffness of the model.} It is only
dependent on the weighting factors and the relative postion of the master nodes.

To get the \textbf{slave} node displacements:

\begin{enumerate}

    \item \textbf{COG} position
    \item $ \m{r}_i $ vectors
    \item compute the weighted average of the master displacements
    \item an unknown rotation is assumed and the displacements on the master
        nodes are written as if they were calculated with a rigid body
        movement with the weighted average displacements and the unknown
        rotations for each node $ i $:

        \begin{equation}
            \m{U}_s = \frac{\sum w_i \m{U}_i}{\sum w_i} + \m{R}_s \times \m{r}_i
        \end{equation}

        where $ \m{U} $ are translational displacements and $ \m{R} $ are rotational
        displacements.

\end{enumerate}


