\newpage
\chapter{Boundary conditions not in GCS}

\section{Skew boundary conditions}
\textit{Skew boundary conditions}, also known as \textbf{Rotated BCs} are implemented
after assembling the global stiffness matrix $ \m{K} $.

Basic equation:

\begin{equation}
    \m{M} \ddot{\m{u}} + \m{C} \dot{\m{u}} + \m{K} \m{u}
    = \m{f}
\end{equation}

First we need to identify all prescribed BCs that do not correspond to the
defined assemblage DOFs and transform the finite equilibrium equations to
correspond to the prescribed BCs. Thus we write:

\begin{equation}\label{skew-bcs}
    \m{u} = \m{T} \closure{\m{u}}
\end{equation}

where $ \closure{\m{u}} $ is the vector of nodal point DOFs in the required
directions. The transformation matrix $ \m{T} $ is an identity matrix
that has been altered by the directional sines and cosines of the components
$ \closure{\m{u}} $ measured in the original displacement directions.
Using \eqref{skew-bcs}, we obtain:

\begin{equation}
    \closure{\m{M}} \closure{\ddot{\m{u}}}
    + \closure{\m{C}} \closure{\dot{\m{u}}}
    + \closure{\m{K}} \closure{\m{u}}
    = \closure{\m{f}}
\end{equation}

where:

\begin{eqarray}\label{skew-eq-of-motion}
    \closure{\m{M}} & = \m{T}^T \m{M} \m{T} \\
    \closure{\m{C}} & = \m{T}^T \m{C} \m{T} \\
    \closure{\m{K}} & = \m{T}^T \m{K} \m{T} \\
    \closure{\m{f}} & = \m{T}^T \m{f}
\end{eqarray}

We should note that the matrix multiplications in \eqref{skew-eq-of-motion}
involve changes only in those columns and rows of $ \m{M} $, $ \m{C} $,
$ \m{K} $ and $ \m{f} $ that are actually affected. In practice,
the transformation is carried out effectively on the element level just prior
to adding the element matrices to the global matrices.


\newpage
\section{Cartesian CSYS}

When a node or an SPC is defined as a \textbf{rotated}/\textbf{skewed} or in
a different type of Coordinate system (e.g. cylindrical) a transformation matrix
needs to be applied to such nodes.

\textbf{Cartesian Coordinate system} definition by angle (2D):
\begin{equation}
    \m{u}_l = \m{T} \m{u}_g
\end{equation}

Unpacked:
\begin{equation}
    \begin{bmatrix}
        u_l \\
        w_l \\
        \psi_l \\
    \end{bmatrix}
    = \begin{bmatrix}
        \phantom{-}\cos \alpha & \sin \alpha & 0 \\
        -\sin \alpha & \cos \alpha & 0 \\
        0 & 0 & 1 \\
    \end{bmatrix}
    \begin{bmatrix}
        u_g \\
        w_g \\
        \psi_g \\
    \end{bmatrix}
\end{equation}

%TODO: \textbf{Cartesian Coordinate system} definition by angle (3D):


\textbf{Cartesian Coordinate system} definition by vectors (3D):

when a \textbf{CSYS} is defined by an \textbf{origin} and \textbf{two vectors}
$ \m{x} = \begin{pmatrix} x_1 & x_2 & x_3 \end{pmatrix} $
and $ \m{y} = \begin{pmatrix} y_1 & y_2 & y_3 \end{pmatrix} $,
then for a Cartesian CSYS it suffices to:
\begin{eqarray}
    \m{\closure{x}} &= \frac{\m{x}}{||\m{x}||} \\
    \m{\closure{y}} &= \frac{\m{y}}{||\m{y}||} \\
    \m{\closure{z}} &= \m{\closure{x}} \times \m{\closure{y}} \\
\end{eqarray}

The transformation relation:
\begin{equation}
    \m{u}_l = \m{T} \m{u}_g
\end{equation}

when unpacked:
\begin{equation}
    \begin{bmatrix}
        u_l \\
        v_l \\
        w_l \\
        \phi_l \\
        \psi_l \\
        \rho_l \\
    \end{bmatrix}
    = \begin{bmatrix}
        \m{x} & \m{0} \\
        \m{y} & \m{0} \\
        \m{z} & \m{0} \\
        \m{0} & \m{x} \\
        \m{0} & \m{y} \\
        \m{0} & \m{z} \\
    \end{bmatrix}
    \begin{bmatrix}
        u_g \\
        v_g \\
        w_g \\
        \phi_g \\
        \psi_g \\
        \rho_g \\
    \end{bmatrix}
\end{equation}

Where $ \m{0} $ is a $ 3 \times 1 $ zero vector.

For a \textbf{2D case} just \textbf{one vector} is enough, the
$ \m{\closure{z}} = \begin{pmatrix} 0 & 0 & 1 \end{pmatrix} $ and following relations
apply:
\begin{eqarray}
    \m{\closure{x}} &= \m{\closure{y}} \times \m{\closure{z}} \\
    \m{\closure{y}} &= \m{\closure{z}} \times \m{\closure{x}} \\
    \m{\closure{z}} &= \m{\closure{x}} \times \m{\closure{y}} \\
\end{eqarray}

This means that a \textbf{CSYS} defined by an \textbf{origin} point
$ \m{O} = \begin{pmatrix} x_O & y_O & z_O \end{pmatrix} $,
point laying on \textbf{x axis}
$ \m{P}_x = \begin{pmatrix} x_{P_x} & y_{P_x} & z_{P_x} \end{pmatrix} $ and a
point laying in \textbf{xy plane}
$ \m{P}_{xy} = \begin{pmatrix} x_{P_{xy}} & y_{P_{xy}} & z_{P_{xy}} \end{pmatrix} $
define a CSYS followingly:
\begin{eqarray}
    \m{x} &= \begin{pmatrix} x_{P_x} - x_O & y_{P_x} - y_O & z_{P_x} - z_O \end{pmatrix} \\
    \m{\closure{x}} &= \frac{\m{x}}{||\m{x}||} \\
    \m{\hat{y}} &= \begin{pmatrix} x_{P_{xy}} - x_O & y_{P_{xy}} - y_O & z_{P_{xy}} - z_O \end{pmatrix} \\
    \m{\closure{\hat{y}}} &= \frac{\m{\hat{y}}}{||\m{\hat{y}}||} \\
    \m{\closure{z}} &= \m{\closure{x}} \times \m{\closure{\hat{y}}} \\
    \m{\closure{y}} &= \m{\closure{z}} \times \m{\closure{x}} \\
\end{eqarray}


\newpage
\section{Cylidrical CSYS}

The basic relation between a point in \textbf{cartesian} and \textbf{cylindrical}
CSYS is (when the cartesian CSYS has the same origin as the cylindrical one
and both CSYS $ z $ axes overlay):
\begin{eqarray}
    x &= r \cos \varphi \\
    y &= r \sin \varphi \\
    z &= z \\
\end{eqarray}

 in one direction and:
 \begin{eqarray}
     r &= \sqrt{x^2 + y^2} \\
     \varphi &= \left\{ \begin{array}{cc}
                indeterminate & if\ x = 0\ and\ y = 0 \\
                \arcsin \frac{y}{r} & if\ x \ge 0 \\
                -\arcsin \frac{y}{r} + \pi & if\ x < 0\ and\ y \ge 0 \\
                -\arcsin \frac{y}{r} + \pi & if\ x < 0\ and\ y < 0 \\
             \end{array} \right.
 \end{eqarray}

 in the other. One can also use the $ \arctan $ function to compute $ \varphi $:
 \begin{equation}
     \varphi = \left\{ \begin{array}{cc}
             indeterminate & if\ x = 0\ and\ y = 0 \\
             \frac{\pi}{2}\frac{y}{|y|} & if\ x = 0\ and\ y \ne 0 \\
             \arctan \frac{y}{x} & if\ x > 0 \\
             \arctan \frac{y}{x} + \pi & if\ x < 0\ and\ y \ge 0 \\
             \arctan \frac{y}{x} - \pi & if\ x < 0\ and\ y < 0 \\
         \end{array} \right.
 \end{equation}

 This is also called the $ \atantwo (y, x) $ function!

 To transform node coordinates from \textbf{cartesian} to \textbf{cylindrical},
 three transforms are in order:

 \begin{enumerate}
     \item translate the coordinates so that \textbf{origin} of the new
         CSYS conforms to the \textbf{origin} of the \textbf{cylidrical}
         CSYS.

         \begin{equation}
             \m{x}_1 = \m{T}_T \m{x}_0
         \end{equation}

         \begin{equation}
             \begin{bmatrix}
                 x_1 \\
                 y_1 \\
                 z_1 \\
                 1 \\
             \end{bmatrix}
             = \begin{bmatrix}
                 1 & 0 & 0 & -x_O^c \\
                 0 & 1 & 0 & -y_O^c \\
                 0 & 0 & 1 & -z_O^c \\
                 0 & 0 & 0 & \phantom{-}1 \\
             \end{bmatrix}
             \begin{bmatrix}
                 x_0 \\
                 y_0 \\
                 z_0 \\
                 1 \\
             \end{bmatrix}
         \end{equation}

         where $ \m{x}_O^c $ denotes the \textbf{cylindrical} CSYS origin
         and $ \m{T}_T $ denotes the translation matrix.

    \item \textbf{rotate} the node to a new \textbf{cartesian} CSYS so that
        the new CSYS \textbf{x-axis} conforms to the cylindrical CSYS
        \textbf{r-axis} and the new CSYS \textbf{z-axis} conforms to the
        cylindrical \textbf{z-axis}.

        \begin{equation}
            \m{x}_2 = \m{T}_R \m{x}_1
        \end{equation}

        When the cylindrical CSYS is defined by three points - \textbf{origin}
         $ \m{x}_O^c = \begin{bmatrix} x_O^C & y_O^c & z_O^c \end{bmatrix}^T $,
         point on \textbf{r-axis} $ \m{x}_{R}^c $
        and a point in \textbf{rz-plane} $ \m{x}_{RZ}^c $, then:

        \begin{equation}
            \vec{\m{r}} = \frac{\m{x}_R^c - \m{x}_O^c}{||\m{x}_R^c - \m{x}_O^c||}
        \end{equation}

        \begin{equation}
            \hat{\m{z}} = \frac{\m{x}_{RZ}^c - \m{x}_O^c}{||\m{x}_{RZ}^c - \m{x}_O^c||}
        \end{equation}

        \begin{equation}
            \vec{\m{y}} = \hat{\m{z}} \times \vec{\m{r}}
        \end{equation}

        \begin{equation}
            \vec{\m{z}} = \vec{\m{r}} \times \vec{\m{y}}
        \end{equation}

        then the transformation can be written:
        \begin{equation}
            \begin{bmatrix}
                x_2 \\
                y_2 \\
                z_2 \\
                1 \\
            \end{bmatrix}
            = \begin{bmatrix}
                \vec{\m{r}}^T & 0 \\
                \vec{\m{y}}^T & 0 \\
                \vec{\m{z}}^T & 0 \\
                \vec{\m{0}}^T & 1 \\
            \end{bmatrix}
            \begin{bmatrix}
                x_1 \\
                y_1 \\
                z_1 \\
                1 \\
            \end{bmatrix}
        \end{equation}

         where $ \vec{\m{0}} = \begin{bmatrix} 0 & 0 & 0 \end{bmatrix}^T $

         Combining $ \m{T}_R $ and $ \m{T}_T $ together:
         \begin{equation}
             \m{T} = \m{T}_R \m{T}_T
         \end{equation}

         \begin{equation}
             \m{T} = \begin{bmatrix}
                 r_1 & r_2 & r_3 & 0 \\
                 y_1 & y_2 & y_3 & 0 \\
                 z_1 & z_2 & z_3 & 0 \\
             \end{bmatrix}
             \begin{bmatrix}
                 1 & 0 & 0 & -x_O^c \\
                 0 & 1 & 0 & -y_O^c \\
                 0 & 0 & 1 & -z_O^c \\
                 0 & 0 & 0 & \phantom{-}1 \\
             \end{bmatrix}
         \end{equation}

         \begin{equation}
             \m{T} = \begin{bmatrix}
                 r_1 & r_2 & r_3 & -r_1 x_O^c -r_2 y_O^c -r_3 z_O^c  \\
                 y_1 & y_2 & y_3 & -y_1 x_O^c -y_2 y_O^c -y_3 z_O^c  \\
                 z_1 & z_2 & z_3 & -z_1 x_O^c -z_2 y_O^c -z_3 z_O^c  \\
             \end{bmatrix}
         \end{equation}


    \item transform the coordinates to the \textbf{cylindrical} CSYS.
        \begin{equation}
            \begin{bmatrix}
                r \\
                \varphi \\
                z \\
            \end{bmatrix}
            = \begin{bmatrix}
                \sqrt{x_2^2 + y_2^2} \\
                \atantwo \left(y_2, x_2\right) \\
                z_2 \\
            \end{bmatrix}
        \end{equation}

\end{enumerate}

\newpage
\begin{bbox}
    A \textbf{cylindrical} coordinate system in FEM in the range of \textbf{small}
    \textbf{displacements} is basically a Cartesian one, just skewed for each
    node separately so that its \textbf{x} axis coincides with \textbf{r} axis.

    Then the cylindrical to global coordinate transformation is such that:
    \begin{equation}
        \begin{bmatrix}
            x \\
            y \\
            z \\
        \end{bmatrix}_{global}
        = \begin{bmatrix}
            r_{11} & r_{12} & r_{13} \\
            r_{21} & r_{22} & r_{23} \\
            r_{31} & r_{32} & r_{33} \\
        \end{bmatrix}
        \begin{bmatrix}
            r \cos \varphi \\
            r \sin \varphi \\
            z \\
        \end{bmatrix}_{local}
        + \begin{bmatrix}
            x \\
            y \\
            z \\
        \end{bmatrix}_{origin}
    \end{equation}

    Therefore the Cartesian CSYS transformation matrix for definition of
    boundary conditions is:

    \begin{enumerate}
        \item first subtract the cylindrical coordinate origin from the global
            coordinates:
            \begin{equation}
                \begin{bmatrix}
                    x \\
                    y \\
                    z \\
                \end{bmatrix}_{local}
                = \begin{bmatrix}
                    x \\
                    y \\
                    z \\
                \end{bmatrix}_{global}
                - \begin{bmatrix}
                    x \\
                    y \\
                    z \\
                \end{bmatrix}_{origin}
            \end{equation}

        \item then create a transformation matrix that maps the global csys
            to rotated one such that the \textbf{r} axis coincides with the
            \textbf{x} axis, $ \M{\varphi} $ axis coincides with the new
            \textbf{y} axis and \textbf{z} is the same:

            \begin{equation}
                \m{r}' = \frac{\m{x}_{local}}{|| \m{x}_{local} ||}
            \end{equation}

            \begin{equation}
                \m{z} = \frac{\m{z}_{cyl}}{|| \m{z}_{cyl} ||}
            \end{equation}

            \begin{equation}
                \M{\varphi} = \m{z} \times \m{r}'
            \end{equation}

            \begin{equation}
                \m{r} = \M{\varphi} \times \m{z}
            \end{equation}

            the transformation matrix is then:
            \begin{equation}
                \m{T} = \begin{bmatrix}
                    r_1 & r_2 & r_3 \\
                    \varphi_1 & \varphi_2 & \varphi_3 \\
                    z_1 & z_2 & z_3 \\
                \end{bmatrix}
            \end{equation}

        \item This transformation matrix $ \m{T} $ is finally applied and
            \textbf{BC}s are imposed.
            \begin{equation}
                \m{T}^T \m{K} \m{T} \m{\closure{u}} = \m{T}^T \m{f}
            \end{equation}

    \end{enumerate}

\end{bbox}

\begin{bbox}
    If there already is a transformation matrix for the cylindrical system
    prepared, such that:
    \begin{equation}
        \m{x}_{local} = \m{T}_{cyl} \left( \m{x}_{global} - \m{x}_{origin} \right)
    \end{equation}

    or:
    \begin{equation}
        \m{x}_{global} = \m{T}_{cyl}^T \m{x}_{local} + \m{x}_{origin}
    \end{equation}

    then:
    \begin{equation}
        r = \sqrt{x_{local}^2 + y_{local}^2}
    \end{equation}

    \begin{equation}
        \varphi = \atantwo \frac{y_{local}}{x_{local}}
    \end{equation}

    \begin{equation}
        z = z_{local}
    \end{equation}

    because $ \cos \varphi = x / r $ and $ \sin \varphi = y / r $,
    then the new transformation matrix is composed:
    \begin{equation}
        \m{T} = \begin{bmatrix}
            \cos & -\sin & 0 \\
            \sin & \cos & 0 \\
            0 & 0 & 1\\
        \end{bmatrix}
        = \begin{bmatrix}
            x / r & -y/r & 0 \\
            y/r & x/r & 0 \\
            0 & 0 & 1 \\
        \end{bmatrix}
    \end{equation}

    and finally displacements are transformed from global to local:
    \begin{equation}
        \m{u}_{local} = \m{T} \m{T}_{cyl} \m{u}_{global}
    \end{equation}
\end{bbox}


