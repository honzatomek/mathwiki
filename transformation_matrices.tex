\documentclass[10pt,b5paper,titlepage]{book}

\usepackage[utf8]{inputenc}
\usepackage{amsmath}
\usepackage{amsfonts}
\usepackage{amssymb}
%\usepackage{xcolor}
\usepackage{color}
\usepackage{graphicx}

\usepackage{hyperref}

\author{Jan Tomek}
\title{\bf Transformation Matrices}

\setlength{\parindent}{0ex}
\renewcommand{\labelitemi}{$\bullet$}
\renewcommand{\labelitemii}{$\bullet$}
\renewcommand{\labelitemiii}{$\bullet$}
\renewcommand{\labelitemiv}{$\bullet$}

%Commands definitions
\newcommand\setbackgroundcolour{\pagecolor[rgb]{0.15,0.15,0.15}}
\newcommand\settextcolour{\color[rgb]{0.9,0.9,0.9}}
\newcommand\invertbackgroundtext{\setbackgroundcolour\settextcolour}

% equal by definition
\newcommand*\eqd{\stackrel{\triangle}{=}}

% {name}[number of arguments][1st default value] etc..
% the argument value is then inserted at #argument_number
\newenvironment{bbox}[1][1.0]
{
    \begin{center}
        \begin{tabular}{|p{#1\textwidth}|}
            \hline\\
}
{
            \\\\\hline
        \end{tabular}
    \end{center}
}

\newenvironment{bboxtitle}[1][1.0]
{
    \begin{center}
        #1\\[1ex]
        \begin{tabular}{|p{#1\textwidth}|}
            \hline\\
}
{
            \\\\\hline
        \end{tabular}
    \end{center}
}

%Command execution.
%If this line is commented, then the appearance remains as usual.
\invertbackgroundtext


\begin{document}

\maketitle

\tableofcontents

\chapter{Coordinate Transformation}

\begin{itemize}
    \item \textbf{Rotation:} \\

        Let $\mathbf{A}^{T} = [a_{x}, a_{y}, a_{z}]$ be the original coordinates,
        $\mathbf{B}^{T} = [b_{x}, b_{y}, b_{z}]$ be the transformed coordinates, and
        $\mathbf{T}$ a tranformation matrix in the form:

        \begin{equation}
            \mathbf{T} = \begin{bmatrix}
                x_1 & x_2 & x_3 \\
                y_1 & y_2 & y_3 \\
                z_1 & z_2 & z_3
            \end{bmatrix} = \begin{bmatrix}
                \mathbf{x} \\
                \mathbf{y} \\
                \mathbf{z}
            \end{bmatrix}
        ,\end{equation}

        where $\mathbf{x} = [x_1, x_2, x_3]$, $\mathbf{y} = [y_1, y_2, y_3]$ and
        $\mathbf{z} = [z_1, z_2, z_3]$ are the \textit{x-axis}, \textit{y-axis} and
        \textit{z-axis} \textbf{unit vectors}, respectively, of the new coordinate
        system defined in the original one, while sharing their origin.

        The following applies:

        \begin{equation}
            \mathbf{T} \times \mathbf{A} = \mathbf{B}
        ,\end{equation}

        and

        \begin{equation}
            \mathbf{T}^{T} \times \mathbf{B} = \mathbf{A}
        .\end{equation}

        \textbf{Note:} Perpendicular vector is  created using \textit{cross-product}.
\end{itemize}


\chapter{Tensor Transformation}

Let $\mathbf{A}$ be the original tensor,
$\mathbf{B}^{T}$ be the transformed tensor, and
$\mathbf{T}$ a tranformation matrix in the form:

\begin{equation}
    \mathbf{A} = \begin{bmatrix}
        \sigma^{A}_{11} & \sigma^{A}_{12} & \sigma^{A}_{13} \\
        \sigma^{A}_{21} & \sigma^{A}_{22} & \sigma^{A}_{23} \\
        \sigma^{A}_{31} & \sigma^{A}_{32} & \sigma^{A}_{33}
    \end{bmatrix}
,\end{equation}

\begin{equation}
    \mathbf{B} = \begin{bmatrix}
        \sigma^{B}_{11} & \sigma^{B}_{12} & \sigma^{B}_{13} \\
        \sigma^{B}_{21} & \sigma^{B}_{22} & \sigma^{B}_{23} \\
        \sigma^{B}_{31} & \sigma^{B}_{32} & \sigma^{B}_{33}
    \end{bmatrix}
,\end{equation}

\begin{equation}
    \mathbf{T} = \begin{bmatrix}
        x_1 & x_2 & x_3 \\
        y_1 & y_2 & y_3 \\
        z_1 & z_2 & z_3
    \end{bmatrix} = \begin{bmatrix}
        \mathbf{x} \\
        \mathbf{y} \\
        \mathbf{z}
    \end{bmatrix}
,\end{equation}

where $\mathbf{x} = [x_1, x_2, x_3]$, $\mathbf{y} = [y_1, y_2, y_3]$ and
$\mathbf{z} = [z_1, z_2, z_3]$ are the \textit{x-axis}, \textit{y-axis} and
\textit{z-axis} \textbf{unit vectors}, respectively, of the new coordinate
system defined in the original one, while sharing their origin.

Then \textbf{tensor transformation} is performed as such:

\begin{equation}
    \mathbf{T} \times \mathbf{A} \times \mathbf{T}^{T} = \mathbf{B}
.\end{equation}

\begin{equation}
    \mathbf{T}^{T} \times \mathbf{B} \times \mathbf{T} = \mathbf{A}
.\end{equation}

\begin{itemize}
    \item \textbf{Principal values:} \\

        The principal values can be found as the \textbf{eigenvalues} of the tensor
        matrix.

        Characteristic equation of tensor $\mathbf{\Sigma}$:

        \begin{equation}
            \mathbf{\Sigma}\mathbf{V} = \mathbf{\Lambda} \mathbf{V}
        ,\end{equation}

        therefore:

        \begin{equation}
            (\mathbf{\Sigma} - \mathbf{\Lambda}\mathbf{I})\mathbf{V} = 0
        ,\end{equation}

        and:

        \begin{equation}
            det(\mathbf{\Sigma} - \mathbf{\Lambda}\mathbf{I}) = 0
        ,\end{equation}

        where $\mathbf{\Lambda}$ is a diagonal matrix of eigenvalues (principal values)
        and $\mathbf{V}$ is the eigenvector matrix.

        The characteristic cubic equation can be also written as:

        \begin{equation}
            \lambda^{3} - I_{1}\lambda^{2} + I_{2}\lambda - I_{3} = 0
        ,\end{equation}

        giving three roots equal to tensor eigenvalues.\\

        First compute \textbf{Invariants} $\mathbf{I}_{n}$:

        \begin{equation}
            \begin{array}{ll}
                I_{1} &= \sigma_{11} + \sigma_{22} + \sigma_{33} \\
                I_{2} &= \sigma_{11}\sigma_{22} + \sigma_{22}\sigma_{33}
                + \sigma_{33}\sigma_{11} - \sigma_{12}^{2} - \sigma_{13}^{2} - \sigma_{23}^{2} \\
                I_{3}
                &= \sigma_{11}\sigma_{22}\sigma_{33}
                - \sigma_{11}\sigma_{23}^{2}
                - \sigma_{22}\sigma_{13}^{2}
                - \sigma_{33}\sigma_{12}^{2}
                + 2 \sigma_{12}\sigma_{13}\sigma_{23}
            \end{array}
        .\end{equation}

        In matrix form:

        \begin{equation}
            \begin{array}{ll}
                I_{1} &= tr[\mathbf{\Sigma}] \\
                I_{2} &= \left| \begin{matrix}
                    \sigma_{11} & \sigma_{12} \\
                    \sigma_{12} & \sigma_{12}
                \end{matrix} \right| + \left| \begin{matrix}
                    \sigma_{11} & \sigma_{13} \\
                    \sigma_{13} & \sigma_{33}
                \end{matrix} \right| + \left| \begin{matrix}
                    \sigma_{22} & \sigma_{23} \\
                    \sigma_{23} & \sigma_{33}
                \end{matrix} \right| \\
                    I_{3} &= det(\mathbf{\Sigma})
            \end{array}
        .\end{equation}

        Then compute \textit{help values}:

        \begin{equation}
            Q = \frac{3 I_{2} - I_{1}^{2}}{9}
        ,\end{equation}

        \begin{equation}
            R = \frac{2 I_{1}^{3} - 9 I_{1} I_{2} + 27 I_{3}}{54} \\
        ,\end{equation}

        \begin{equation}
            \theta = cos^{-1} \left( \frac{ R }{ \sqrt{- Q^{3}}} \right)
        .\end{equation}

        Lastly to get \textbf{principal values}:

        \begin{equation}
            \overline{\sigma_{1}}
            = 2 \sqrt{-Q} \cos{\left(\frac{\theta}{3}\right)} + \frac{1}{3}I_{1}
        .\end{equation}

        \begin{equation}
            \overline{\sigma_{2}}
            = 2 \sqrt{-Q} \cos{\left(\frac{\theta + 2 \pi}{3}\right)} + \frac{1}{3}I_{1}
        .\end{equation}

        \begin{equation}
            \overline{\sigma_{3}}
            = 2 \sqrt{-Q} \cos{\left(\frac{\theta + 4 \pi}{3}\right)} + \frac{1}{3}I_{1}
        .\end{equation}

        The \textbf{principal values are not sorted}. The principal tensor is then:

        \begin{equation}
            \mathbf{\Sigma} = \begin{bmatrix}
                \sigma_{1} & 0 & 0 \\
                0 & \sigma_{2}  & 0 \\
                0 & 0 & \sigma_{3}
            \end{bmatrix}
        ,\end{equation}

        where $\overline{\sigma}_{1}$, $\overline{\sigma}_{2}$ and $\overline{\sigma}_{3}$
        $\implies$ $\sigma_1 > \sigma_2 > \sigma_3$.

        Afterwards the \textbf{principal axes} are obtained as:

        \begin{equation}
            \begin{array}{l}
                \mathbf{\Gamma}_{1} = \mathbf{\Sigma} - \sigma_{1}\mathbf{I} \\
                \mathbf{\Gamma}_{2} = \mathbf{\Sigma} - \sigma_{2}\mathbf{I} \\
                \mathbf{\Gamma}_{3} = \mathbf{\Sigma} - \sigma_{3}\mathbf{I}
            \end{array}
        ,\end{equation}

        where $\mathbf{\Gamma}$ is a coordinate system corresponding to the
        principal value \textit{i} and $\mathbf{I}$ is a $3 \times 3$
        identity matrix.

        The vectors corresponding to each \textbf{principal value} are obtained:

        \begin{equation}
            \mathbf{\Gamma}_{1}
            = \begin{bmatrix}
                \overline{x}_{11} & \overline{x}_{12} & \overline{x}_{13}\\
                \overline{y}_{11} & \overline{y}_{12} & \overline{y}_{13}\\
                \overline{z}_{11} & \overline{z}_{12} & \overline{z}_{13}
            \end{bmatrix}
            = \begin{bmatrix}
                \overline{\mathbf{x}}_{1}\\
                \overline{\mathbf{y}}_{1}\\
                \overline{\mathbf{z}}_{1}
            \end{bmatrix}
        ,\end{equation}
        \begin{equation}
            \mathbf{\Gamma}_{2}
            = \begin{bmatrix}
                \overline{x}_{21} & \overline{x}_{22} & \overline{x}_{23}\\
                \overline{y}_{21} & \overline{y}_{22} & \overline{y}_{23}\\
                \overline{z}_{21} & \overline{z}_{22} & \overline{z}_{23}
            \end{bmatrix}
            = \begin{bmatrix}
                \overline{\mathbf{x}}_{2}\\
                \overline{\mathbf{y}}_{2}\\
                \overline{\mathbf{z}}_{2}
            \end{bmatrix} \\
        ,\end{equation}
        \begin{equation}
            \mathbf{\Gamma}_{3}
            = \begin{bmatrix}
                \overline{x}_{31} & \overline{x}_{32} & \overline{x}_{33}\\
                \overline{y}_{31} & \overline{y}_{32} & \overline{y}_{33}\\
                \overline{z}_{31} & \overline{z}_{32} & \overline{z}_{33}
            \end{bmatrix}
            = \begin{bmatrix}
                \overline{\mathbf{x}}_{3}\\
                \overline{\mathbf{y}}_{3}\\
                \overline{\mathbf{z}}_{3}
            \end{bmatrix}
        ,\end{equation}

        where $\overline{\mathbf{x}}$, $\overline{\mathbf{y}}$ and $\overline{\mathbf{z}}$ are vectors of size 3.

        \begin{equation}
              \mathbf{x} = \frac{\overline{\mathbf{y}}_{1} \times \overline{\mathbf{z}}_{1}}
              {|\overline{\mathbf{y}}_{1} \times \overline{\mathbf{z}}_{1}|}\\
        ,\end{equation}
        \begin{equation}
              \mathbf{y} = \frac{\overline{\mathbf{z}}_{2} \times \overline{\mathbf{x}}_{2}}
              {|\overline{\mathbf{z}}_{2} \times \overline{\mathbf{x}}_{2}|}\\
        ,\end{equation}
        \begin{equation}
              \mathbf{z} = \frac{\overline{\mathbf{x}}_{3} \times \overline{\mathbf{y}}_{3}}
              {|\overline{\mathbf{x}}_{3} \times \overline{\mathbf{y}}_{3}|}
        .\end{equation}

        The principal axes are also the \textbf{eigenvectors} of tensor $\mathbf{\Sigma}$:

        \begin{equation}
            \mathbf{V} = \begin{bmatrix}
                \mathbf{x} & \mathbf{y} & \mathbf{z}
            \end{bmatrix} = \begin{bmatrix}
                x_1 & y_1 & z_1 \\
                x_2 & y_2 & z_2 \\
                x_3 & y_3 & z_3
            \end{bmatrix}
        .\end{equation}

    \item \textbf{Maximum Shear value:}\\

        From: \url{https://wp.optics.arizona.edu/optomech/wp-content/uploads/sites/53/2016/10/OPTI_222_W21.pdf}\\
        From: \url{https://www.continuummechanics.org/principalstressesandstrains.html}\\
        From: \url{https://www.ecourses.ou.edu/cgi-bin/eBook.cgi?doc=&topic=me&chap_sec=07.2&page=theory} \\

        Maximum \textbf{shear stress} occurs at an angle of 45 degrees to principal axes.
        If principal stresses are aligned such that $\sigma_1 > \sigma_2 > \sigma_3$
        and the stress tensor is:

        \begin{equation}
            \mathbf{\Sigma} = \begin{bmatrix}
                \sigma_1 & 0 & 0 \\
                0 & \sigma_2 & 0 \\
                0 & 0 & \sigma_3
            \end{bmatrix}
        ,\end{equation}

        then maximum shear stress $\tau_{max}$ can be obtained as follows:

        \begin{equation}
            \tau_{max} = \frac{\sigma_1 - \sigma_3}{2}
        \end{equation}

        and is in at 45 degrees angle in the 1-3 plane of the principal axes to the
        principal coordinate system.

        More generally (in 2D):

        \begin{equation}
            \tau_{max} = \sqrt{\left(\frac{\sigma_{x} - \sigma_{y}}{2}\right)^{2}
            + \tau_{xy}^{2}}
        .\end{equation}

        \begin{bbox}
            When the principal tensor is rotated by 45 degrees to obtain \textbf{maximum
            shear stress}, the axial stress components corresponding to this shear
            stress are equal.
        \end{bbox}\\

    \item \textbf{Example:}

        Let a stress tensor in principal coordinate system be:

        \begin{equation}
             \mathbf{\Sigma} = \begin{bmatrix}
                 433 & 0 & 0 \\
                 0 & 125 & 0 \\
                 0 & 0 & 24
             \end{bmatrix}
        ,\end{equation}

        then $\sigma_1 = \sigma_{max} = 433 \text{ MPa}$ and
        $\sigma_3 = \sigma_{min} = 24 \text{ MPa}$.\\

        Transformation matrix for rotating by 45 degrees in 3-1 plane is:

        \begin{equation}
            \mathbf{T} = \begin{bmatrix}
                \frac{\sqrt{2}}{2} & 0 & -\frac{\sqrt{2}}{2} \\
                0 & 1 & 0 \\
                \frac{\sqrt{2}}{2} & 0 & \frac{\sqrt{2}}{2}
            \end{bmatrix}
        .\end{equation}

        Rotating tensor $\mathbf{\Sigma}$ by transformation matrix $\mathbf{T}$ we get:

        \begin{equation}
            \mathbf{\Sigma}_{\tau} = \mathbf{T}\mathbf{\Sigma}\mathbf{T}^{T}
            = \begin{bmatrix}
                228.5 & 0 & 204.5 \\
                0 & 125 & 0 \\
                204.5 & 0 & 228.5
            \end{bmatrix}
        ,\end{equation}

        where:

        \begin{equation}
            \begin{array}{l}
                \tau_{max} = \sigma_{13} = \sigma_{31} = 204.5 \\
                \sigma_{11} = \sigma_{33} = 228.5 \\
                \sigma_{22} = \sigma_2
            \end{array}
        .\end{equation}

        Check for corrext values:

        \begin{equation}
            \tau_{max} = \frac{\sigma_1 - \sigma_3}{2} = \frac{433 - 24}{2}
            = \frac{409}{2} = 204.5 \text{ MPa} \implies \text{OK!}
        .\end{equation}

        $\sigma_2$ has not changed as it is colinear with the axis of rotation.

        The relationship between principal normal streses and maximum shear stresses
        can be better understood by examining a plot of the stresses as a function
        of the rotation angle.

        Notice that there are multiple $\theta_{p}$ and $\theta_{\tau - max}$
        angles because of the periodical nature of the equations. However, they
        will give the same absolute values.

        At the principal stress angle, $\theta_{p}$, the shear stress will always be zero,
        as show on the diagram. And the maximum shear stress will occur when the
        two principal normal stresses, $\sigma_1$ and $\sigma_2$, are equal.

        \begin{figure}[h]
            \centering
            \includegraphics[width=0.8\textwidth]{img/stresses_as_function_of_angle}
            \caption{Stresses as a function of angle}
            \label{fig:stresses_as_function_of_angle-png}
        \end{figure}

        \begin{bbox}
            When $\sigma_{x}$ or $\sigma_{y}$ are either max or min, the shear stress
            $\tau_{xy}$ is equal to 0. When shear stress $\tau_{xy}$ is max or min,
            the stresses $\sigma_{x}$ and $\sigma_{y}$ are equal.
        \end{bbox}


\end{itemize}


\chapter{Moment of Inertia}

Moment of Inertia = \textbf{tensor}.

\begin{itemize}
    \item \textbf{Steiner's share:}\\

        \begin{equation}
            \mathbf{I}_{ss} = m \times \begin{bmatrix}
                y^{2} + z^{2} & -xy & -xz \\
                -yz & x^{2} + z^{2} & -yz \\
                -zx & -zy & x^{2} + y^{2}
            \end{bmatrix}
        .\end{equation}

        \textbf{Identities for a skew-symmetric matrix}\\

        In order to compare formulations of the parallel axis theorem using
        skew-symmetric matrices and the tensor formulation, the following
        identities are useful.

        Let $[\mathbf{R}]$ be the skew-symmetric matrix associated with the
        position vector $\mathbf{R} = (x, y, z)$, then the prodictuct in
        the inertia matrix becomes:

        \begin{equation}
            -[\mathbf{R}][\mathbf{R}] = \begin{bmatrix}
                0 & -z & y \\
                z & 0 & -x \\
                -y & x & 0
            \end{bmatrix}^{2}
            = \begin{bmatrix}
                y^{2} + z^{2} & -xy & -xz \\
                -yz & x^{2} + z^{2} & -yz \\
                -zx & -zy & x^{2} + y^{2}
            \end{bmatrix}
        .\end{equation}

        This product can be computed using the matrix formed by the outer product
        $[\mathbf{R} \mathbf{R}^{T}]$ using the identity

        \begin{equation}
            \begin{array}{ll}
                 \\
                -[\mathbf{R}]^{2}
                &= |\mathbf{R}|^{2}|\mathbf{E}_{3}| - [\mathbf{R}\mathbf{R}^{T}] \\
                &= \begin{bmatrix}
                    x^{2} + y^{2} + z^{2} & 0 & 0 \\
                    0 & x^{2} + y^{2} + z^{2} & 0 \\
                    0 & 0 & 0
                \end{bmatrix} - \begin{bmatrix}
                    x^{2} & xy & xz \\
                    yx & y^{2} & yz \\
                    zx & zy & z^{2}
                \end{bmatrix}
            \end{array}
        ,\end{equation}

        where $[\mathbf{E}_{3}$ is the $n \times n$  identity matrix.

        Also notice, that

        \begin{equation}
            |\mathbf{R}|^{2} = \mathbf{R} \cdot \mathbf{R} = tr [\mathbf{R}\mathbf{R}^{T}]
        .\end{equation}

        where $tr$ denotes the sum of the diagonal elements of the outer product
        matrix, known as its \textit{trace}.

\end{itemize}

\end{document}
