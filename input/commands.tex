% python code format
\usemintedstyle{native}
\setminted[python]{breaklines, framesep=2mm, fontsize=\footnotesize, numbersep=5pt}
\newminted[python]{python}{linenos=true,
                           frame=lines,
                           baselinestretch=1.2,
                           mathescape,
                           xleftmargin=0cm,
                           framesep=2mm,
                           fontsize=\footnotesize}

\setlength{\parindent}{0ex}
\setlength{\parskip}{1em plus 0.1em minus 0.2em}
\renewcommand{\labelitemi}{$\bullet$}
\renewcommand{\labelitemii}{$\bullet$}
\renewcommand{\labelitemiii}{$\bullet$}
\renewcommand{\labelitemiv}{$\bullet$}

%Commands definitions
\newcommand\setbackgroundcolour{\pagecolor[rgb]{0.15,0.15,0.15}}
\newcommand\settextcolour{\color[rgb]{0.9,0.9,0.9}}
\newcommand\invertbackgroundtext{\setbackgroundcolour\settextcolour}

\newcommand*{\Scale}[2][4]{\scalebox{#1}{$#2$}}%
\newcommand*{\Resize}[2]{\resizebox{#1}{!}{$#2$}}%

% Lagrangian symbol
\newcommand{\lagr}{\mathop{\mathcal{L}}}
\DeclareMathOperator{\Lagr}{\mathcal{L}}

% matrix notation
\newcommand{\m}{\mathbf}
% matrix notation greek letters
\newcommand{\M}{\pmb}

% null symbol
\newcommand{\Null}{\text{\O}}

% atan2 math symbol
\DeclareMathOperator{\atantwo}{atan2}
\DeclareMathOperator{\arctantwo}{arctan2}

\newcommand{\closure}[2][3]{%
{}\mkern#1mu\overline{\mkern-#1mu#2}}

% maximum number of entries in matrix
\setcounter{MaxMatrixCols}{20}

% equal by definition
\newcommand*\eqd{\stackrel{\triangle}{=}}

% +=
\newcommand*\eqp{\stackrel{+}{=}}

% {name}[number of arguments][1st default value] etc..
% the argument value is then inserted at #argument_number
\newenvironment{bbox}[1][0.96]
{
    \begin{center}
        \begin{tabular}{|p{#1\textwidth}|}
            \hline\\
}
{
            \\\\\hline
        \end{tabular}
    \end{center}
}

\newenvironment{bboxtitle}[1][0.96]
{
    \begin{center}
        #1\\[1ex]
        \begin{tabular}{|p{#1\textwidth}|}
            \hline\\
}
{
            \\\\\hline
        \end{tabular}
    \end{center}
}

\newcount\myloopcounter
\newcommand{\repeatit}[3][10]{%
    \myloopcounter1% initialize the loop counter
    \loop\ifnum\myloopcounter < #1
    #2#3%
    \advance\myloopcounter by 1%
    \repeat% start again
    #2%
}

\newenvironment{qbox}
{
%\centering{\huge{?}}
\begin{center}
    \repeatit[42]{?}{\ }
\end{center}
%\hrule
}
{
%\hrule
\begin{center}
    \repeatit[42]{?`}{\ }
\end{center}
}

\newenvironment{eqarray}
{
    \begin{eqnarray}
        \begin{aligned}
}
{
        \end{aligned}
    \end{eqnarray}
}

